% -----------------------------------------------------------------------------
% \subsection{Geant4 Tasks Performance Analysis on Titan}
% \label{ssec:panda_titan}

Currently, 20 instances of PanDA Broker are deployed on a total of 4 DTNs, 5
instances for each DTN. Each broker controls from 15 to 300 detector simulation
jobs per submission, for a theoretical maximum concurrent use of 96,000
cores. Since November 2015, the PanDA Brokers have operated only in backfill
mode, without a defined time allocation, and running at the lowest priority on
Titan.

Figure~\ref{fig:core-hours-utilization} shows Titan core hours used by ATLAS
from January 2016 to February 2017. During that period, ATLAS consumed a total
of 73.8 million Titan core hours, for an average of 7M hours a month, with a
minimum of 3.3M hours in April 2016 and a maximum 14.8M hours in February 2017.
On February, PanDA Brokers used almost twice as much backfill availability than
in any other month. This is likely due to the hardware upgrades made to the
DTNs. The absence of continuous monitoring of the DTNs does not allow for the
specific identification of the bottleneck but spot measurements of the DTNs load
indicate that a faster CPU and better networking were likely responsible for the
improved performance. No relevant code update was made between January and
February 2017 and PanDA Brokers logs indicated that they were able to respond
more promptly to backfill availability.

\begin{figure}[htp]
    \subfloat[]{
        \includegraphics[clip,width=\columnwidth]{figures/cpu_hours.png}
        \label{fig:core-hours-utilization}
    }

    \subfloat[]{
        \includegraphics[clip,width=\columnwidth]{figures/backfill_consumption.png}
        \label{fig:backfill-utilization}
    }
\caption{(a).(b).}
\end{figure}

Investigations of the CPU load of the upgraded DTNs shows an average
XX\mtnote{ask Danila for data} load. This indicates that further hardware
upgrades would not be likely to improve significantly the performance of the
PanDA Brokers. Nonetheless, the current load suggests that the number of brokers
per DTN could be increased. This would enable the submission of a larger number
of concurrent jobs to the Titan's PBS queue, allowing for PanDA to consume a
higher percentage of the overall backfill availability.

Figure~\ref{fig:backfill-utilization} shows backfill utilization efficiency,
defined as the fraction of Titan’s total cores available via backfill utilized
by ATLAS, from January 2016 to February 2017. ATLAS reached 18\% average
efficiency with a minimum 8.9\% efficiency on April 2016 and a maximum 33.5\%
efficiency on February 2017. The number of total backfill cores available in
April 2016 was 38.1M and in February 2017 33.1M. This shows that the improvement
in the efficiency was determine by the different in total backfill availability.

Figure~\ref{fig:hpc-workload-utilization} shows that between January 2016 and
February 2017, about XX million detector simulation jobs were completed on
Titan, for a total of XXXM events processed. This is equivalent to XX\% of the
total amount of detector simulations performed by ATLAS in the same period of
time, and XX\% of the total number of events processed. Comparatively, Titan
contributed between XX\% and XX\% of the total WLCG availability. When
accounting for the amount of unused backfill availability and the rate of
improvement of PanDA efficiency, these figures confirms the fundamental role
that supercomputers' resources can play for the LHC Run 3.

\begin{figure}[htp]
    \includegraphics[clip,width=\columnwidth]{figures/cpu_hours.png}
\caption{\mtnote{Placeholder for the diagram/data I asked to Danila and
Sergey}}
\label{fig:hpc-workload-utilization}
\end{figure}

The scalability of PanDA Broker depends on the DTNs' resources. Each PanDA
Broker can execute 1 job on up to 300 work nodes for a total of 4,800 cores.
Scaling above this threshold requires instantiating multiple PanDA Brokers on
the same DTN and, once the DTN resources are saturated, on multiple DTNs. In
principle, this is not a problem on Titan as DTNs are supposed to be available
and can be periodically upgraded. Assuming the efficiency reached on February as
reference, using all the available backfill resources on Titan will require to
instantiate around 60 PanDA Brokers. It is not clear whether this is within the
scope of the current DTN capabilities, especially when considering the increased
staging in and out of files from the DTNs and ATLAS Grid sites.

The current design and architecture of the PanDA Broker proven to be very
robust. The failure rate of jobs produced by issues with PanDA Broker and PanDA
in general is XX\%. This confirms the benefits given by reusing most of the code
base of the PanDA Pilot and adopting RADICAL-SAGA for communicating with Titan's
PBS batch system and monitoring jobs execution. During XX months of executing
production jobs, most of the failures were due to: \ldots, \ldots, and \ldots.
The usual corrective measure were put in place and the progressive decline of
the failure rate confirms that PanDA Broker reached production-grade stability
on Titan.\mtnote{I requested figures about failure types and rate to Sergey and
Danila.}

\begin{itemize}
    \item Comparison hardware performance Titan/Grid
    \item problem with lustre addressed with ramdisk
    \item Summary overall figure of performance analogous to those produced with NGE. Diagram as discussed in F2F meeting at Rutgers.
    \item Competition for cache in AMD processors.
\end{itemize}

Currently, two main parameters measure the performance of the detector
simulation jobs submitted to Titan: (i) the time taken to setup
AthenaMP~\cite{aad2010atlas}, the ATLAS software framework integrating the
GEANT4 simulation toolkit; and (ii) the distribution of the time taken by the
Geant4 toolkit to simulate a certain number of events.

AthenaMP has an initialization and configuration stage. At initialization time,
AthenaMP is  assembled from a large number of shared libraries. Once
bootstrapped, every algorithm and service is configured by a set of Python
scripts. Both these operations result in a large number of read operations,
including those required to  access of small python scripts.


% AthenaMP is a multipurpose framework that needs to be configured depending on
% the type of payalod that needs to be executed. For Geant4, this configuration
% process links up to 200 libraries, all requiring filesystem read and write
% operations.

% The execution of Geant4 is mostly compute-intensive requiring to write around
% XXKB per 100 events on disk and no more than 2GB of memory.

% -----------------------------------------------------------------------------
% \subsection{PanDA Shared Library I/O Performance Impact at OLCF}

% Athena, the ATLAS framework has a configuration and initialization stage. At
% this stage, the running job is assembled on the fly from a large number of
% shared libraries. Also, at this stage, every algorithm and service is being
% configured, at run time, by a corresponding set of Python scripts, which
% results in a large number of read operations accesses to small python
% scripts, with many includes and imports Python calls.

Originally, all shared libraries of AthenaMP and the python scripts for job
configuration were stored on the Spider 2 Lustre file system. However, the I/O
patterns of the initialization and configuration stages degraded the performance
of the filesystem. Since Spider 2 is a center-wide file system, shared by all
OLCF resources and users, this resulted in lower overall interactive and
metadata performance at OLCF\@. \mtnote{I am afraid the details about the trace
are too specific given the space constraints of a SC submission. Please feel
free to uncomment it if you disagree.}
% As can be seen in Listing~\ref{mdstrace} the metadata I/O activity for ATLAS
% exhibits a spike corresponding to the beginning of the runs before tapering
% off.

% \begin{minipage}{\linewidth}
% \begin{lstlisting}[language=bash,frame=single,basicstyle=\ttfamily\tiny,caption=ATLAS metadata trace,label=mdstrace]
% XK7 Application 9205593
%       39012 RPCs from 300 of 300 nodes
%         ~69291.96 per sec
%           37851 LDLM_ENQUEUE RPCs    ~67229.83 per sec
%                 pmin 13us pavg 42us pmax 4983us
%           611 LDLM_CANCEL RPCs    ~1085.24 per sec
%                 pmin 10us pavg 16us pmax 32us
%           277 MDS_CLOSE RPCs    ~492.00 per sec
%                 pmin 15us pavg 20us pmax 38us
%           154 MDS_READPAGE RPCs    ~273.53 per sec
%                 pmin 170us pavg 292us pmax 671us
%           86 MDS_GETXATTR RPCs    ~152.75 per sec
%                 pmin 15us pavg 21us pmax 65us
%           30 MDS_GETATTR RPCs    ~53.29 per sec
%                 pmin 16us pavg 20us pmax 27us
%           3 MDS_REINT RPCs    ~5.33 per sec
%                 pmin 103us pavg 136us pmax 196us
%           Overall times
%                 pmin 10us pavg 42us pmax 4983us
%
% XK7 Application 9205355
%       8698 RPCs from 62 of 62 nodes
%         ~15449.13 per sec
%           8445 LDLM_ENQUEUE RPCs    ~14999.76 per sec
%                 pmin 16us pavg 41us pmax 780us
%           92 MDS_CLOSE RPCs    ~163.41 per sec
%                 pmin 15us pavg 23us pmax 52us
%           55 MDS_READPAGE RPCs    ~97.69 per sec
%                 pmin 189us pavg 291us pmax 534us
%           52 LDLM_CANCEL RPCs    ~92.36 per sec
%                 pmin 12us pavg 16us pmax 25us
%           41 MDS_GETXATTR RPCs    ~72.82 per sec
%                 pmin 16us pavg 20us pmax 27us
%           13 MDS_GETATTR RPCs    ~23.09 per sec
%                 pmin 16us pavg 21us pmax 38us
%           Overall times
%                 pmin 12us pavg 42us pmax 780us
%
% \end{lstlisting}
% \end{minipage}

% As this problem was identified, the OLCF staff has enabled read-only access to
% certain NFS-exported directories from Titan compute nodes. This in turn
% allowed the OLCF staff to install a software package from a Titan login node
% and have it available read-only on a Titan compute node.

This issue was addressed by moving the AthenaMP distribution to a read-only NFS
directory made accessible from the Titan work nodes. This eliminated the
metadata contention problem improving metadata read performance of two orders of
magnitude: from \~6,300 seconds on Lustre to \~1,500 seconds on NFS. Further, at
configuration time, the input files of each job were stored to a ramdisk on the
work nodes. This offered three orders of magnitude performance improvement in
the file validation step: from 1,320 seconds on Lustre to 40 seconds on NFS.

%Figure~\ref{fig:atlas-perf-improvement} shows the overall ATLAS performance
%improvement on Titan. The circled region illustrates the switch from Lustre to
%NFS-exported directory for hosting the ATLAS release.

%\begin{figure}[!htb]
%    \centering
%    \begin{tabular}{cc}
%        {\includegraphics[width=0.48\textwidth]{figures/panda-completed-jobs-sw-move.pdf}}\\
%    \end{tabular}
%    \caption{ATLAS performance improvement on Titan. The circled region shows the switch from Lustre to NFS-exported directory for hosting the ATLAS release.}
%\label{fig:atlas-perf-improvement}
%\end{figure}


% -----------------------------------------------------------------------------
\subsection{PanDA I/O Impact at OLCF}

To better understand the I/O impact of ATLAS PanDA project on Titan
supercomputing environment we analyzed 1,175 jobs ran on the week of 10/25/2016,
for a total of 174 hours. Table~\ref{panda-olcf-stats} shows the overall
statistical breakdown of the observed file I/O impact of ATLAS at OLCF\@.
Figures~\ref{fig:atlas-titan-io-read} and~\ref{fig:atlas-titan-io-written} show
the file read and write I/O histograms for these 1,175 jobs, respectively.
Figures~\ref{fig:atlas-titan-file-open} and~\ref{fig:atlas-titan-file-close}
show the file $open()$ and $close()$ metadata load histograms of the same 1,175
ATLAS jobs, respectively.

As can be seen from Table~\ref{panda-olcf-stats}, the number of nodes used by
ATLAS jobs vary between 1 and 300, while the average is at 35. 75\% of the ATLAS
jobs consume less than 25 and 92\% consume less than or equal to 100 Titan
compute nodes. During the 174 hours of data collection, we observed that 6.75
ATLAS jobs were executed on average per hour on Titan and ran for 1.74 hours on
average.

ATLAS jobs issues a large number of file read operations, as can be seen in
Table~\ref{panda-olcf-stats}. The maximum amount read by any ATLAS job in
aggregate in this observed period was less than 250 GB and the maximum amount
written in aggregate was less than 75 GB\@. The average amount read per job is
20 GB and average amount written is 6 GB\@.

\begin{figure}[!htb]
    \centering
    \begin{tabular}{cc}
        {\includegraphics[width=0.48\textwidth]{figures/panda_data_read_finer_hist.pdf}}\\
    \end{tabular}
    \caption{ATLAS file read operation histogram on Titan for week of 10/25/16.}
\label{fig:atlas-titan-io-read}
\end{figure}

\begin{figure}[!htb]
    \centering
    \begin{tabular}{cc}
        {\includegraphics[width=0.48\textwidth]{figures/panda_data_written_finer_hist.pdf}}\\
    \end{tabular}
    \caption{ATLAS file write I/O histogram on Titan for week of 10/25/16.}
\label{fig:atlas-titan-io-written}
\end{figure}

Per job read and write file I/O statistics show an interesting pattern.
Table~\ref{panda-olcf-stats} indicates that the amount of data read per ATLAS
compute node on Titan is less than 400 MB on average, while the amount of data
written per node is less than 170 MB on average. This correlates with our
finding that ATLAS PanDA jobs are read heavy. However, as can be seen in
Table~\ref{panda-olcf-stats} and figures {fig:atlas-titan-io-read} and
{fig:atlas-titan-io-written}, the distribution between read and written amount
of data per job is quite different from one another. The read operation
distribution per job shows a long tail, ranging from 12.5 GB to 250 GB, while
the written amount of data has a very narrow distribution.

\begin{figure}[!htb]
    \centering
    \begin{tabular}{cc}
        {\includegraphics[width=0.48\textwidth]{figures/panda_file_open_hist.pdf}}\\
    \end{tabular}
    \caption{ATLAS file $open()$ histogram on Titan for week of 10/25/16.}
\label{fig:atlas-titan-file-open}
\end{figure}

\begin{figure}[!htb]
    \centering
    \begin{tabular}{cc}
        {\includegraphics[width=0.48\textwidth]{figures/panda_file_close_hist.pdf}}\\
    \end{tabular}
    \caption{ATLAS file $close()$ histogram on Titan for week of 10/25/16.}
\label{fig:atlas-titan-file-close}
\end{figure}

On the metadata I/O breakdown, ATLAS PanDA jobs yield 23 file $open()$
operations and 5 file $close()$ operations per second. The file $open()$
operations listed here don't include the file $stat()$ operations. As can be
seen from figures~\ref{fig:atlas-titan-file-open}
and~\ref{fig:atlas-titan-file-close}, they exhibit similar distributions. The
maximum number of file $open()$ operations are around 170 on average and the
maximum number of file $close()$ operations are 39 on average per second per
job. The total number of file $open()$ operations is 172,089,760 for the
observed window of 174 hours, while the total number of file $close()$
operations stand at 40,132,992 for the total 1,175 ATLAS PanDA jobs in the same
observation window. The difference between these two values is puzzling and it
is under investigation at the time of writing this paper. One possible
explanation is that ATLAS PanDA jobs perhaps don't call a file $close()$
operation per every file $open()$ issued.

\begin{table*}[t]
\centering
\begin{tabular}{lllllllll}
 & Num. Nodes & Duration (s) & Read (GB) & Written (GB) & GB Read/nodes & GB Written/nodes & $open()$ & $close()$ \\
Min & 1 & 1,932 & 0.01 & 0.03 & 0.00037 & 0.02485 & 1,368 & 349 \\
Max & 300 & 7,452 & 241.06 & 71.71 & 0.81670 & 0.23903 & 1,260,185 & 294,908 \\
Average & 35.66 & 6,280.82 & 20.36 & 6.87 & 0.38354 & 0.16794 & 146,459.37 & 34,155.74 \\
Std. Dev. & 55.33 & 520.99 & 43.90 & 12.33 & 0.19379 & 0.03376 & 231,346.55 & 53,799.08
\end{tabular}
\caption{The Statistical breakdown of the I/O impact of 1,175 PanDA jobs executed at OLCF for the week of 10/25/16}
\label{panda-olcf-stats}
\end{table*}

Overall, based on our experiments with real-world jobs, it can be safely
concluded that the file and metadata I/O load of ATLAS PanDA project on the OLCF
Titan supercomputing environment and the Spider 2 file system is not detrimental
to the center operations and the overall impact is minimal, at the current scale
of the project.
