\ifreview
Comments to address in this section:
\begin{itemize}
	\color{red} 
	\item REVIEWER 2: the topic of other HPC usage models [OCCAM,FRHPC],
	elastic simulation jobs [ELASTICMC], opportunistic HPC usage [ATLASHPC] or
	proxy infrastructure [ACTATLAS,ACTHYDRA] is not discussed.\mtnote{I looked
	for these models and found few references to them. All in all, I am not
	seeing how these fit within the background of pilot-enabled workload
	managers.}\jhanote{As you've exercised diligence in checking, we should
	ignore these suggestions (which are at best advisory, not mandatory)}
\end{itemize}
\fi

Several pilot-enabled WMS were developed for the LHC experiments:
AliEn~\cite{Bagnasco2010} for ALICE\@; DIRAC~\cite{Paterson2010} for LHCb;
GlideinWMS~\cite{sfiligoi2008glideinwms} for CMS\@; and
PanDA~\cite{maeno2014evolution} for ATLAS\@. These systems implement similar
design and architectural principles: centralization of task and resource
management, and of monitoring and accounting; distribution of task execution
across multiple sites; unification of the application interface; hiding of
resource heterogeneity; and collection of static and sometimes dynamic
information about resources.

AliEn, DIRAC, GlideinWMS and PanDA all share a similar design with two types
of components: the management ones facing the application layer and
centralizing the capabilities required to acquire tasks' descriptions and
matching them to resource capabilities; and resource components used to
acquire compute and data resources and information about their capabilities.
Architecturally, the management components include one or more queue and a
scheduler that coordinates with the resource modules via push/pull protocols.
All resource components include middleware-specific APIs to request for
resources, and a pilot capable of pulling tasks from the management modules
and executing them on its resources.

%~\cite{thain2005distributed}

AliEn, DIRAC, GlideinWMS and PanDA also have similar implementations. These
WMS were initially implemented to use Grid resources using the Condor
software ecosystem~\cite{thain2005distributed}. Accordingly, all LHC WMS
implemented Grid-like authentication and authorization systems and adopted a
computational model based on distributing a large amount of single/few-cores
tasks across hundreds of sites.

All LHC experiments produce and process large amounts  of data from actual
collisions in the accelerator and from their simulations. Dedicated,
multi-tiered data systems have been built to store, replicate, and
distributed these data. All LHC WMS interface with these systems to move data
to the sites where related compute tasks are executed or to schedule compute
tasks where (large amount of) data are already stored.

It is interesting to note that most WMS developed to support LHC experiments
are gradually evolving towards integrating HPC resources, though none have
reached sustained operational usage at the scales that PanDA has achieved for
ATLAS on Titan.