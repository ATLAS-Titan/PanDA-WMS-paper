% As seen in \S\ref{ssec:panda_titan}, PanDA Broker is deployed on the DTNs
% as Titan's worker nodes lack connectivity to the wide area network.
The lack of pilot capabilities in PanDA Broker impacts both the efficiency
and the flexibility of PanDA's execution process. Pilots could improve
efficiency by increasing throughput and enabling greater backfill
utilization. Further, pilots would make it easier to support heterogeneous
workloads.

The absence of pilots 
% imposes the static coupling between MPI scripts submitted to the PBS batch
% system and detector simulations (\S\ref{sec:panda_titan}). This
makes the scheduling of multiple generations of workload on the same PBS job
impossible: once a statically defined number of detector simulations are
packaged into a PBS job and this job is queued on Titan, no further
simulations can be added to that job. New simulations have to be packaged
into a new PBS job that needs to be submitted to Titan based upon backfill
and PanDA Brokers availability.

The support of multiple generations of workload would enable more efficient
use of the backfill availability walltime. Currently, when a set of
simulations ends, the PBS job also ends, independent of whether more
wall-time would still be available. With a pilot, additional simulations
could be executed to utilize all the available wall-time, while avoiding
further job packaging and submission overheads.

Multiple generations would also relax two assumptions of the current execution
model: knowing the number of simulations before submitting the MPI script, and
having a fixed number of events per simulation (currently 100). Pilots would
enable the scheduling of simulations independently from whether they were
available at the moment of submitting the pilot. Further, simulations with a
varying number of events could be scheduled on a pilot, depending on the
amount of remaining walltime and the distribution of execution time per event,
as shown in \S\ref{ssec:panda_titan}, Fig.~\ref{fig:comparison-8-16cores}.
These capabilities would increase the efficiency of the PanDA Broker when
there is a large difference between the number of cores and walltime.

Pilots can offer a payload-independent scheduling interface while hiding the
mechanics of coordination and communication among multiple worker nodes. This
could eliminate the need for packaging payload into MPI scripts within the
broker, greatly simplifying the submission process. This simplification would
also enable the submission of different types of payload, without having to
develop a specific PBS script for each payload. The submission process would
also be MPI-independent, as MPI is used for coordination among multiple
worker nodes, not by the payload.

% ---------------------------------------------------------------------------
\subsection{Implementation}
\label{sec:arch}

The implementation of pilot capabilities within the PanDA Broker require
quantification of the effective benefits that it could yield and, on the base
of this analysis, a dedicated engineering effort. We developed a prototype of
a pilot system capable of executing on Titan to study experimentally the
quantitative and qualitative benefits that it could bring to PanDA\@. We
called this prototype Next Generation Executor (NGE).

NGE is a runtime system to execute heterogeneous and dynamically determined
tasks that constitute workloads. Fig.~\ref{fig:arch-overview} illustrates its
current architecture as deployed on Titan: the two management modules (Pilot
and Unit) represent a simplified version of the PanDA Broker while the agent
module is the pilot submitted to Titan and executed on its worker nodes. The
communication between PanDA Broker and Server is abstracted away as it is not
immediately useful to evaluate the performance and capabilities of a pilot on
Titan.

\begin{figure}
  \centering
   \includegraphics[width=\columnwidth]{figures/rp_architecture_compact_atlaswms_paper.pdf}
   \vspace{-0.3in}
  \caption{NGE Architecture: The PilotManager and UnitManager reside on a DTN
  while the Agent is executed on a worker node. Color coding: gray for
  entities external to NGE\@; white for APIs; purple for NGE's modules; green
  for pilots; yellow for module's components.}
\label{fig:arch-overview}
\end{figure}

NGE exposes an API to describe workloads (Fig.~\ref{fig:arch-overview}, green
squares) and pilots (Fig.~\ref{fig:arch-overview}, red circles), and to
instantiate a PilotManager and a UnitManager. The PilotManager submits pilots
to Titan's PBS batch system via SAGA API (Fig.~\ref{fig:arch-overview}, dash
arrow). %, as done by PanDA Broker to submit MPI scripts. 
Once scheduled, the pilot's Agent is bootstrapped on a MOM node and the
Agent's Executors on the worker nodes.

The UnitManager and the Agent communicate via a database instantiated on a
DTN so as to be reachable by both modules. The UnitManager schedules units to
the Agent's Scheduler (Fig.~\ref{fig:arch-overview}, solid arrow) and the
Agent's Scheduler schedules the units on one or more Agent's Executor.
% for execution. The Agent's executors can manage one or more worker nodes,
% depending on performance evaluations.

The Agent uses the Open Run-Time Environment (ORTE) for communication and
coordination of the execution of units. This environment is a critical
component of the OpenMPI implementation~\cite{castain05:_open_rte}.
%, cug-2016}.
ORTE is able to minimize the system overhead while submitting tasks by
avoiding filesystem bottlenecks and race conditions with network sockets.