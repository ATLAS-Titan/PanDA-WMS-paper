As seen in \S\ref{sec:panda_deployment}, the absence of wide area network
connectivity on Titan's worker nodes was one of the main reasons for PanDA
Broker not to implement pilots. As discussed in \S\ref{sec:panda_titan}, the
absence of pilots imposes the static coupling between PBS jobs and detector
simulations, and development tailored to each type of payload that needs to be
executed.

The static coupling between PBS jobs and detector simulations makes impossible
to schedule multiple generations of workload on the same PBS job. Specifically,
once a number of detector simulations are packaged into a PBS job and this job
is queued on Titan, no further simulations can be added to that job. New
simulations have to be packaged into a new PBS job that need to be submitted to
Titan on the base of the backfill availability of that moment.

The support of workload generations would enable a more efficient use of the
backfill availability walltime. Currently, when a set of simulations ends also
the PBS job ends, independently on whether more walltime would still be
available. With a pilot, more simulations could be executed so to utilize all
the available walltime, while avoiding further job packaging and submission
overheads.\mtnote{Are there more reasons for wanting multiple generations?}

Multiple generations would also help with the transition from an execution model
based on a fixed amount of event per simulation (100 at the moment) to one in
which the number of events is decided on the base of the available walltime.
Pilots would enable to package simulations with a homogeneous number of events,
especially when core and walltime availabilities would diverge as when a small
amount of cores is available for a long amount of time or \textit{vice versa}.
In this situations, a PBS job would would require all simulation to have either
a large or a small number of events; a pilot would instead support multiple
generations of simulations with a different number of events and therefore
duration.

Pilots can offer a payload-independent scheduling interface while hiding the
mechanics of coordination and communication among multiple worker nodes. This
could eliminate the need for packaging payload into PBS jobs within the broker,
greatly simplifying the submission process. This simplification would also
enable the submission of different types of payload, without having to develop a
specific PBS script for each payload. The submission process would also be
MPI-independent, as MPI is used for coordination among multiple worker nodes,
not by the payload.

\mtnote{I tried to argue that pilots facilitates the submission to a normal
queue but I was not able to find a convincing argumentation. When we consider
only Titan, a PBS job is the only way to submit a job to its queue(s). With NGE,
we translate a pilot description to a PBS job and then submit it to a queue via
a PBS command. PanDA Brokers do something very similar: they package detector
simulations into a PBS job and they submit it via a PBS command. The fact that
PanDA Brokers define the number of cores and walltime of a PBS job on the base
of backfill information, does not make the packaging or the submission processes
any different: PanDA Brokers are already submitting to the nomal queue. If I got
it right, PanDA Brokers could already submit a job with, say, 12000 cores and 2
hours walltime. They don't do it because they have no allocation and they would
wait a long time in the queue (maybe, let's see what our experiments say about
that). I do not see how NGE could make all this any different.}

\mtnote{NOTE: I would speak about about backfill/not backfill in the discussion
of NGE, when speaking about its generality towards resources, i.e., unified
submission and scheduling process across different resources and multiple
resources. Part of this generality is being able to submit to whatever batch
system is supported by SAGA (including Titan's PBS and all its queues) and, in
case, to multiple resources at the same time.}

\mtnote{NOTE: NGE is not able to use backfill on Titan as we do not have
specific functionalities to interrogate the Moab scheduler about backfill
availability. This is why NGE should be presented as the pilot for PanDA Broker
and not as an alternative to PanDA Broker. Further, NGE alone would not be able
to speak directly to PanDA Server.}

% -----------------------------------------------------------------------------
\subsection{Next Generation Executer}
\label{ssec:nge}

\begin{enumerate}
    \item Rationale and design (why)
    \item Architecture (how)
    \item Integration
    \item Characterization (experiments)
\end{enumerate}

We are developing a prototype of Next Generation Executer (NGE) to add pilot
capabilities to the PanDA Broker. NGE is \ldots
