As seen in \S\ref{sec:panda_deployment}, PanDA Broker has been designed to
submit PBS jobs to Titan's batch system instead of pilots due to the absence of
wide area network connectivity on Titan's work nodes. As discussed in
\S\ref{sec:panda_titan}, the unavailability of pilots imposes three main
limitations: (i) static coupling between PBS jobs and detector simulations and
(ii) support for a single type of payload.

The static coupling between PBS jobs and detector simulations makes impossible
to schedule multiple generations of workload on the same PBS job. Specifically,
once a number of detector simulations are packaged into a PBS job and this job
is queued on Titan, no further simulations can be added to that job. New
simulations have to be packaged into a new PBS job that need to be submitted to
Titan on the base of the backfill availability of that moment.

The support of workload generations would enable a more efficient use of the
backfill availability walltime. Currently, when a set of simulations ends also
the PBS ends, independently on whether more walltime would still be available.
With a pilot, more simulations could be executed so to utilize all the walltime
while avoiding further job packaging and submission overheads.\mtnote{more?}

Pilots offer a general scheduling interface while hiding the mechanics of
coordinating resources across multiple work nodes. This eliminates the need of
packaging payload into PBS jobs within the broker, greatly simplifying the
submission process. This simplification would make it possible to submit
different types of payload without having to develop a specific PBS script for
every payload.
