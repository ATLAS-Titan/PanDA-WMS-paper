As seen in \S\ref{sec:panda_deployment}, the absence of wide area network
connectivity on Titan's work nodes was one of the main reasons for PanDA Broker
not to implement pilots. As discussed in \S\ref{sec:panda_titan}, the absence of
pilots imposes the static coupling between PBS jobs and detector simulations,
and development tailored to each type of payload that needs to be executed.

The static coupling between PBS jobs and detector simulations makes impossible
to schedule multiple generations of workload on the same PBS job. Specifically,
once a number of detector simulations are packaged into a PBS job and this job
is queued on Titan, no further simulations can be added to that job. New
simulations have to be packaged into a new PBS job that need to be submitted to
Titan on the base of the backfill availability of that moment.

The support of workload generations would enable a more efficient use of the
backfill availability walltime. Currently, when a set of simulations ends also
the PBS job ends, independently on whether more walltime would still be
available. With a pilot, more simulations could be executed so to utilize all
the available walltime, while avoiding further job packaging and submission
overheads.\mtnote{Are there more reasons for wanting multiple generations?}

Pilots can offer a payload-independent scheduling interface while hiding the
mechanics of coordination and communication among multiple work nodes. This
could eliminate the need for packaging payload into PBS jobs within the broker,
greatly simplifying the submission process. This simplification would also
enable the submission of different types of payload, without having to develop a
specific PBS script for every payload. Finally, the submission process would be
MPI-independent, as MPI is used for coordination among multiple work nodes, not
by the payload.

\mtnote{NOTE: I do not mention backfill availability Vs normal availability
(i.e., submitting to the queue a `normal' job) because, in theory, the design of
PanDA Broker enables this capability. I would speak about about backfill/not
backfill in the discussion of NGE, when speaking about its generality. Part of
this generality is being able to submit to whatever batch system is supported by
SAGA, including Titan's PBS and all its queues.}

\mtnote{NOTE: NGE is not able to use backfill on Titan as we do not have specific functionalities to interrogate the Moab scheduler about backfill availability. This is why NGE should be presented as the pilot for PanDA Broker and not as an alternative to PanDA Broker. NGE alone would not be able to speak directly to PanDA Server.}

% -----------------------------------------------------------------------------
\subsection{Next Generation Executer}
\label{ssec:nge}

\begin{enumerate}
    \item Rationale and design (why)
    \item Architecture (how)
    \item Integration
    \item Characterization (experiments)
\end{enumerate}

We are developing a prototype of Next Generation Executer (NGE) to add pilot
capabilities to the PanDA Broker. NGE is \ldots
