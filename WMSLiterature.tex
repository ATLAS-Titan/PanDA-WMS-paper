The Production and Distributed Analysis (PanDA) is a Workload Management System
(WMS) developed to support the operations of the LHC ATLAS experiment. PanDA is
specifically design to manage the execution of distributed workloads and
workflows via pilots.

WMS for distributed executions are a type of middlware, designed to coordinate
specific activities about resources and workloads: discovering and selecting
resources, submitting the tasks of a workload, and monitoring the  execution of
those tasks~\cite{marco2009glite}. Pilot is an abstraction that enables
multi-level scheduling by decoupling resource acquisition from workload
scheduling~\cite{turilli2015comprehensive}. Pilots are implemented by resource
placeholders: a request for a certain amount of resources is scheduled on a site
(e.g., by submitting a job or requesting a virtual machine) and once the
resources are acquired, tasks are scheduled directly to the pilot, not to site's
scheduler.

WMS can implement the pilot abstraction: WMS control the acquisition of
resources by means of pilots and then manage the execution of workloads or
workflows' tasks on those pilots. In tis way, pilots expose a uniform scheduling
interface and execution environment, isolating the differences among the
heterogeneous resources of multiple sites; WMS centralize the management of the
execution process, offering a unified interface to the application layer.
Pilot-enabled WMS are particularly relevant for LHC experiments, where large
amount of heterogeneous resources, distributed across multiple sites have to be
coordinated to execute tens of million of tasks per months while producing and
storing petabytes of data.

Several pilot-enabled WMS were developed for the LHC experiments:
AliEn~\cite{Bagnasco2010} for ALICE; DIRAC~\cite{Paterson2010} for LHCb;
GlideinWMS~\cite{sfiligoi2008glideinwms} for CMS; and
PanDA~\cite{maeno2014evolution} for ATLAS. These systems implement similar
design and architectural principles: separation of concern between management
and execution, data and compute; centralization of the management capabilities;
distributed execution of the execution capabilities across multiple sites;
unification of the application interface; hiding of resource heterogeneity;
centralized monitoring and accounting of both resources and execution; and
collection of static and sometimes dynamic information about resources.

For example, \mtnote{Summarize design and architecture descriptions of the four
WMS stressing some of their main similarities. Use a subset of the `principles'
listed above.}

The implementations of pilot-enabled WMS mainly differ in how specialized they
are for diverse middleware used to provide resources, and for specific
workflows. All the WMS developed for LHC assume to use sites exposing different
flavors of Grid middleware. As such, all the WMS implementations support
Grid-like authentication and authorization and a computational model based on
distributing a large amount of single-core tasks across hundreds of
sites\mtnote{Is this true?}.

% Differences among workflow supported.

AliEn, DIRAC, and PanDA are implemented to support specific experimental
workflows, while GlideinWMS was developed as component of the Condor software
framework. ALICE workload is \ldots. LHCb workload is \ldots. CMS workload is
\ldots.

The ATLAS project requires data, compute, and memory-intensive workflows, both
for single users and so called `production' research groups. Generally, PanDA
was designed to support mostly single-core tasks, requiring a variable \ldots.

% In this section, we present other workload manager systems that have been
% developed to support LHC related but they have been used also to support other
% computing demanding projects.

%\begin{itemize}
%   \item \emph{Clients}: consist in a set of APIs that allows users to submit
% job requests. Clients interact directly with DIRAC central services.
%   \item \emph{Resources}: they can be PC's, site clusters and Grids. Agents
% interact with them without distinction.
%\end{itemize}

% -----------------------------------------------------------------------------
\subsubsection{Alien}

Alien is a Workload and Data Management system composed of a set of middleware
tools and services entirely based on web-services and standard protocols. The
framework was originally developed for the ALICE experiment~\cite{Alice1995} but
subsequently used by several virtual organizations
\cite{McClatechey2003,GPCALMA}.

% The system has been deployed in 2001 for distributed production of Monte Carlo
% data, detector simulation and reconstruction.

Alien is composed of two type of services~\cite{Bagnasco2010}:

\begin{itemize}
    \item \emph{Central services}, these services are unique for each virtual organization, therefore there is only one configuration point for the management;
    \item \emph{Site services}, they provide the interfacing to local resources and Grid services running on a VO-box;
\end{itemize}

The most important Central services are:
\begin{itemize}
    \item \emph{Task Queue}, a database that keeps track of all the tasks submitted to the system and their current execution status;
    \item \emph{Brokers}, they are the core of task executions and data transfers; they receive tasks in form of JDL,  keep them ordered by priority and send them to the CE for execution;
    \item \emph{Optimizers}, they are used to minimize the work of the Broker by scanning periodially the task queue and re-arranging the tasks in such a way that fairness and priority policies are guaranted;
    \item \emph{Data Catalogue}, it keeps track of the scripts and files uploaded on Storage Elements.
\end{itemize}

%The Computing Agents are instead site services that monitor the local
% Computing Element, advertise site's capabilities and are responsible for
% submitting the JobAgents.

% Information about the status of the sites and central services, full job
% statistics and monitoring information are kept in a MonALISA repository.
%% CLUSTER MONITOR SHOULD BE EQUAL TO COMPUTING AGENT
%% Job Manager should be equal to TaskQueue
%% Process Monitor == PIlot????

The task execution in Alien is usually distributed over several sites. Each of these sites has at least one service called ClusterMonitor. On one side the Cluster Monitor is used to communicate with Central services (Task queue and Broker), on the other side it can manage Computing Elements (CE) by starting and stopping them whenever it receives the signal.

The CE is in charge of the execution of the tasks on the resources. A CE usually
is associated with a batch queue and can send the tasks to the worker nodes
controlled by the queue. The CE asks the Broker for tasks to execute by sending
its description in form of Job Description Language (JDL). Once received the
JDL, the Broker will try to match it with the JDL of the tasks in queue. If a
match exists then the Broker sends the tasks JDL to the CE. Immediately after
receiving a job JDL, the CE creates a new service on the worker node called
ProcessMonitor. This service allows the CE (and the rest of Alien services
through the CE) to interact with the job while is running \cite{Saiz2003}. This
execution strategy is called ``pull mode'' due to the fact that CE asks for
tasks.

% Since Alien exploits JDL, a workload can be described task by task according
% to features such as: task priorities, the level of parallelism (one core,
% multi-core, MPI etc..) and also the DCR that should be targeted for the
% execution.

% Job submission is implemented by following the so-called ``pull mode'' which
% is composed of the following steps:

% \begin{enumerate}
%   \item the VO-Box monitors the status of the site queues through polls to
% the resource running on the CE;
%   \item the Job Broker receives a report everytime slots become available;
%   \item if the Task Queue is not empty, the Job Broker asks the VO-box to
% submit a number of Agents;
%   \item finally, the JobAgents are submitted  to the site Computing Element
% either by way of that sends them back to the site Computing Element or,
% wherever available, directly through the CREAM interface on the CE itself.
% \end{enumerate}

% -----------------------------------------------------------------------------
\subsubsection{DIRAC}

DIRAC (Distributed Infrastructure with Remote Agent Control) Workload and Data
Management System has been developed within the CERN LHCb project to manage the
processing of detector data, Monte Carlo simulations, and end-user analyses
\cite{Tsaregorodtsev2004}. DIRAC's architecture relies on two entities
\cite{Paterson2010}:

\begin{itemize}
    \item \emph{Services}: serve Clients and Agents by performing crucial
    operations such as Task Management, Configuration, Bookkeeping and
    Accounting.
    \item \emph{Agents}: perform repetitive actions like querying file
    catalogs, monitoring of jobs on resources.
\end{itemize}

In the same way of Alien, DIRAC implements a pull scheduling. Furthermore DIRAC
was the first WMS to exploit the concept of Pilot Agent on the Grid
\cite{Casajus2010}, i.e., gLite jobs that are submitted to the grid when tasks
arrive into the WMS. DIRAC pilot system has four main logical components:

\begin{itemize}
    \item a set of TaskQueues that collect tasks submitted by users, multiple
    TaskQueue being created depending on the requirements and ownership of the
    tasks;
    \item a set of JobWrappers that are executed on the DCR to bind compute
    resources and execute tasks submitted by the users;
    \item a set of TaskQueueDirectors that submits JobWrappers to target DCRs;
    \item a MatchMaker that matches requests from JobWrappers to suitable tasks
    into TaskQueues.
\end{itemize}

The DIRAC execution model can be summarized in five steps: 1. a user submits its
workload in form of tasks to the WMS Job Manager; 2. submitted tasks are
validated and added to a new or an existing TaskQueue, depending on the task
properties; 3. TaskQueueDirector evaluates TaskQueues and a suitable number of
JobWrappers are submitted to available DCRs; 4. JobWrappers get instantiated on
the DCRs and, then, ask for tasks to the MatchMaker; 5. JobWrappers execute
tasks while JobWrapper’s Watchdog monitor them.

TaskQueueDirectors deploy Pilots by getting a list of TaskQueues and calculating
the number of pilot to submit according to user priorities. Once deployed on the
compute resource, Pilots, a.k.a. JobWrappers, hold the resource in the form of
single or multiple cores, spanning portions, whole, or multiple compute nodes.
Pilots do not expose data capabilities although the system allows the user to
perform both data staging and data replication. TaskQueues, TaskQueueDirectors,
and the MatchMaker are implemented as services whereas the JobWrapper is
implemented within the Agents together with the WatchDog.

% JDL and \emph{Transformation Management System} (TMS) are used for task
% description \cite{Tsaregorodtsev2006,Pacini2006}. The latter allows the user
% to describe complex data dependences.

% -----------------------------------------------------------------------------
\subsubsection{HTCondor Glidein and GlideinWMS}

The HTCondor Glidein system as part of the HTCondor software ecosystem. The
HTCondor Glidein is a pilot based system to aggregate DCRs with heterogeneous
middleware into HTCondor resource pools. Condor is based on daemons
collaborating by exchanging messages over the network. We can isolate four main
logical components~\cite{Sfiligoi2008}:

\begin{itemize}
    \item \emph{Schedd}, implements a queuing system that holds workload tasks;
    \item \emph{Startd}, controls the DCR resources.
    \item \emph{Collector}, holds references to all the activeSchedd/Startd daemons;
    \item \emph{Negotiator} matches tasks queued in a Schedd to resources handled by a Startd.
\end{itemize}

Glidein-WMS has been developed to integrate HTCondor Glidein to  automate the
deployment and management of Glideins on multiple types of DCR middleware. The
integration required three additional logical components:

\begin{itemize}
    \item \emph{Glidein Factories} that submit tasks to the DCRs middleware;
    \item a set of \emph{Virtual Organizations (VO) Frontend} daemons that
    match the tasks on one or more Schedd to the resource attributes;
    \item a \emph{Collector} that holds references to all the active Glidein
    Factories and VO Frontend daemons.
\end{itemize}

The execution model of the HTCondor Glidein system can be summarized in nine
steps: 1. the user submits a Glidein (i.e., a job) to a DCR batch scheduler; 2.
once executed, this Glidein bootstraps a Startd daemon; 3. the Startd daemon
advertises itself with the Collector; 4. the user submits the tasks of the
workload to the Schedd daemon; 5. the Schedd advertises these tasks to the
Collector; 6. the Negotiator matches the requirements of the tasks to the
properties of one of the available Startd daemon (i.e., a Glidein); 7. the
Negotiator communicates the match to the Schedd; 8. the Schedd submits the tasks
to the Startd daemon indicated by the Negotiator; 9. the task is executed.

By using GlideinWMS, the user does not have to submit Glidein directly but only
tasks to Schedd. From there: 1. every Schedd advertises its tasks with the VO
Frontend; 2. the VO Frontend matches the tasks’ requirements to the resource
properties advertised by the WMS Connector; 3. the VO Frontend places requests
for Glideins instantiation to the WMS Collector; 4. the WMS Collector contacts
the appropriate Glidein Factory to execute the requested Glideins; 5. the
requested Glideins become active on the DCRs; and 6. the Glideins advertise
their availability to the (HTCondor) Collector. From there on the execution
model is the same as described for the HTCondor Glidein Service.

% The resources managed by a single Glidein (i.e., pilot) are limited to compute
% resources. Glideins may bind one or more cores, depending on the target DCRs.
% For example, heterogeneous HTCondor pools with resources for desktops,
% workstations, small campus clusters, and some larger clusters will run mostly
% single core Glideins. More specialized pools that hold, for example, only DCRs
% with HTC, Grid, or Cloud middleware may instantiate Glideins with a larger
% number of cores. Both HTCondor Glidein and GlideinWMS provide abstractions for
% file staging but pilots are not used to hold data or network resources. The
% process of pilot deployment is the main difference between HTCondor Glidein
% and GlideinWMS. While the
%
% HTCondor Glidein system requires users to submit the pilots to the DCRs,
% GlideinWMS automates and optimizes pilot provisioning. GlideinWMS attempts to
% maximize the throughput of task execution by continuously instantiating
% Glideins until the queues of the available Schedd are emptied. Once all the
% tasks have been executed, the remaining Glideins are terminated. HTCondor
% Glidein and GlideWMS expose the interfaces of HTCondor to the application
% layer and no theoretical limitations are posed on the type and complexity of
% the workloads that can be executed. For example, DAGMan (Directed Acyclic
% Graph Manager) has been designed to execute workflows by submitting tasks to
% Schedd, and a tool is available to design applications based on the
% master-worker coordination pattern.
%
% Both HTCondor Glidein and GlideWMS rely on one or more HTCondor Collectors to
% match task requirements and resource properties, represented as ClassAds. This
% matching can be evaluated right before the scheduling of the task. In this
% way, late binding is achieved but early binding remains unsupported.
%
% \begin{table*}
% \begin{center}
% \begin{tabular}{llllllll}
%   \hline
%     Pilot System  &Logical Components& Execution Strategy & Binding  & Workload Definition  &  Broker  & \\
%   \hline
%     Alien & Central services, site services & pull, pilot via Co-pilot & Late & JDL & Condor (ClassAd attributes) &\\
%     DIRAC & Services, Agents& pull, pilot-based & Late & JDL, WF (TMS) & Condor(ClassAd attributes) &\\
%     glidein WMS& Daemons & pull, pilot-based  & Late & Pegaus, DAGMan & Condor(ClassAd attributes) &\\
%   \hline
% \end{tabular}
% \end{center}
% \caption{Comparison of the three Workload and Data Management Systems}\label{tab:Summary}
% \end{table*}
