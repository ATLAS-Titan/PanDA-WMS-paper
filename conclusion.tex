\ifreview
Comments to address in this section:
\begin{enumerate}
	\item REVIEWER 3: How extensible is this? Do you have a sense of backfill
	availability on other systems? Is this unique to Titan or Cray or more
	ubiquitous of an option?
\end{enumerate}
\fi

The deployment of PanDA Broker on Titan enabled distributed computing on a
leadership-class HPC machine at unprecedented scale. In the past 13 months,
PanDA WMS has consumed almost 52M core-hours on Titan, simulating 3.5\% of
the total number of detector events of the ATLAS production Monte Carlo
workflow. We described the implementation and execution process of PanDA WMS
(\S\ref{sec:panda_overview}) and PanDA Broker (\S\ref{sec:panda_titan}),
showing how they support and enable distributed computing at this scale on
Titan, a leadership-class HPC machine managed by OCLF.

We characterized the experience by evaluating the efficiency, scalability and
reliability of both PanDA Broker and AthenaMP as deployed on Titan
(\S\ref{sec:panda_titan}). Our characterization highlighted the strengths and
limitations of the current design and implementation: PanDA Brokers enable
the sustained execution of millions of simulations per week but further work
is required to optimize its efficiency and reliability
(\S\ref{ssec:broker_titan}). PanDA Brokers support the concurrent execution
of multiple AthenaMP instances, enabling each AthenaMP to perform the
concurrent execution of up to 16 Geant4 simulators. Nonetheless, our
characterization showed how improving I/O performance could reduce overheads
(\S\ref{ssec:athenamp_titan}), increasing the overall utilization of Titan's
backfill availability.

% More generally, this paper highlights the fundamental role of WMS for
% experimental science.

HEP was amongst the first, if not the first experimental community to realize
the importance of using WMS to manage their computational campaign(s). As
computing becomes increasingly critical for a range of experiments, the
experience foreshadows the importance of WMS for other experiments (such as
SKA, LSST etc.).  These experiments will have their own workload
characteristics, resources types and federation constraints, as well metrics
of performance. The experience captured in this paper will prove invaluable
for designing WMS for computational campaigns and will provide a baseline to
evaluate the relative merits of different approaches.

The 52M core hours used by ATLAS, via PanDA, is over 2\% of the total
utilization on Titan over the same period, bringing the time-averaged
utilization of Titan to be consistently upwards of 90\%. Given that the
average average utilization of most other leadership class machines is less
(e.g., NSF's flagship Blue Waters the average utilization fluctuates between
60--80\% (see XDMoD\cite{bw-sucks})) there is ample headroom for similar
approaches elsewhere. These unprecedented efficiency gains aside, this work
is just a starting point towards more effective operational models for future
leadership and online analytical platforms~\cite{foap-url}. These platforms
will have to support ever increasing complex workloads with varying models
for dynamic resource federation.

% Recent efforts with Google Cloud for the CMS experiment --
% HEPCloud~\cite{hepcloud,googlehep} represent similar yet different
% approaches to dynamic resource federation. As the type, heterogeneity and
% complexity of future platforms  (e.g., Exascale, Commercial Clouds) for
% science increases, there will be a greater need and emphasis on
% abstractions to ensure WMS can both utilize as well as support evolving
% platforms. Our experience demonstrated the advantages arising from an
% abstractions based approach \textendash{} in particular scalable
% implementations of pilot abstraction.
