\ifreview
Comments to address in this section:
\begin{enumerate}
    \color{red} 
    \item REVIEWER 3: The scalability benchmarks in section "V.B Experiments"
    should be better explained: what *extacly* is being kept constant in each
    experiment, and what is being varied and over what range. Also the
    captions in the figures do not help much.
\end{enumerate}
\fi

% ---------------------------------------------------------------------------
\subsection{Experiments}\label{sec:ngeExp}

We designed experiments to characterize the performance of the NGE on Titan,
with an emphasis on understanding its overhead and thus the cost of
introducing new functionalities. We perform three groups of experiments in
which we investigate the weak scalability, weak scalability with multiple
generation, and strong scalability of the NGE\@.

Each experiment entails executing multiple instances of AthenaMP to simulate
a pre-determined number of events. All the experiments have been performed by
configuring AthenaMP to use all the 16 cores of Titan's worker nodes.

We measured the execution time of the pilots and of AthenaMP within them,
collecting timestamps at all stages of the execution. Experiments were
performed by submitting pilots to Titan's batch queue. The turnaround
time of an individual run is determined by queue waiting times. Since we are
interested only in the performances of the NGE, we removed queue time from
our statistics.

% ---------------------------------------------------------------------------
\subsubsection{Weak scalability}

In this experiment we run as many AthenaMP instances (hereafter referred to
as tasks) as the number of nodes controlled by the pilot. Each AthenaMP
simulates 100 events, requiring \(\sim 4200\) seconds on average.

Tasks do not wait within the Agent's queue since one node is available to
each AthenaMP instance. Task execution  overheads result primarily from
three factors: (i) the initial bootstrapping of the pilot on the nodes; (ii)
the UnitManager's dispatching of units (tasks) to the agent; and (iii) time
for the agent to bootstrap all the tasks on the nodes.

We tested pilots with 250, 500, 1000 and 2000 worker nodes and 2 hours
walltime. The time duration is determined by the Titan's walltime policy.
Fig.~\ref{fig:weakScal1a} depicts the average pilot duration, the average
execution time of AthenaMP, and the pilot overhead as function of the pilot
size.

\begin{figure}[!t]
    \includegraphics[height=4.5cm,width=\columnwidth]{./figures/NGE/weak1.pdf}
   	\vspace{-0.3in}
    \caption{Weak scalability: average pilot duration, average duration of
    one AthenaMP execution, and pilot's overhead as a function of pilot sizes
    (200, 500, 1000 and 2000 nodes).}\label{fig:weakScal1a}
\end{figure}

We observe that, despite some fluctuations due to external factors (e.g.,
Titan's shared filesystem and the shared database used by the NGE), the
average execution time of AthenaMP ranges between 4500 and 4800 seconds. We
also observe that in all the cases the gap between AthenaMP execution times
and the pilot durations is minimal, although it slightly increases with the
pilot size. We notice that NGE's overhead grows linearly with the number
of units.

% ---------------------------------------------------------------------------
\subsubsection{Weak scalability with multiple generation}

The NGE provides an important new capability of submitting multiple
generations of tasks to the same pilot. In order to investigate the cost
of doing so, we performed a variant of the weak scalability experiments. This
stresses the pilot's components, as new tasks are scheduled for execution on
the Agent while other tasks are still running.

In these experiments, we run five AthenaMP instances per node. As these
experiments are designed to investigate the overhead of  scheduling and
bootstraping of AthenaMP instances, the number of events simulated by each
AthenaMP task was reduced to sixteen such that the running time of each
AthenaMP was $\sim 1200$ seconds on average. This experiment design choice
does not affect the objectives or accuracy of the experiments, but allows us
to scale experiments to large node counts by conserving project allocation.

We ran pilots with 256, 512, 1024 and 2048 worker nodes and 3 hours walltime.
Fig.~\ref{fig:weakScal2a} depicts the average pilot duration, the average
execution time of five sequential generations of AthenaMP, and the
corresponding overhead. The difference between the two durations is more
marked than in the previous experiments. Despite this, we notice that the
growth of the overhead is consistent with the increment of the number of tasks
per node for pilots with 256, 512 and 1024 worker nodes, and less than linear
for the pilot with 2048 worker nodes.

\begin{figure}[!t]
    \includegraphics[height=4.5cm,width=\columnwidth]{./figures/NGE/weak2.pdf}
    \vspace{-0.3in}
    \caption{Weak scalability with multiple generations (where each
    generation has approximately 1/6th the number of events compared to Fig.
    7): average pilot duration, average duration of sequential AthenaMP
    executions, and pilot's overhead for pilot with 256, 512, 1024 and 2048
    nodes.}\label{fig:weakScal2a}
\end{figure}

%\vspace{-0.2in}

% ---------------------------------------------------------------------------
\subsubsection{Strong scalability}

We investigate strong scalability by running the same number of tasks for
different pilot sizes. We used 2048 AthenaMP instances and pilots with 256,
512, 1024 and 2048 nodes. Thus, the number of AthenaMP generations is equal
to eight times the size of the smallest pilot and corresponds to the size of
the largest pilot. As a consequence, the number of consecutive generations of
AthenaMP decreases with the pilot size. These experiments are designed to
investigate whether pilot overhead is affected by the degree of concurrency
within the pilot and/or the number of tasks. Each AthenaMP instance simulates
sixteen events as in the previous experiment.

Fig.~\ref{fig:strongScala} shows the average pilot duration and the average
execution time of possibly sequential AthenaMP instances. We notice that the
difference between the pilot duration and the AthenaMP execution times is
almost constant for all the pilot sizes, although the overall duration of the
pilot decreases linearly with the pilot size.

\begin{figure}[!t]
    \includegraphics[height=4.5cm,width=\columnwidth]{./figures/NGE/strong.pdf}
    \vspace{-0.3in}
    \caption{Strong scalability: Average pilot duration, average duration of
    sequential AthenaMP executions, and pilot's overhead for pilots with 256,
    512, 1024 and 2048 nodes.}\label{fig:strongScala}
\end{figure}

\vspace{-0.05in}
