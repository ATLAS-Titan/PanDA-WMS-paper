\documentclass[10pt, conference, compsocconf]{IEEEtran}

\input{head}
\input{include}

\usepackage{listings}
\usepackage{comment}

% Some very useful LaTeX packages include:
% (uncomment the ones you want to load)

% *** MISC UTILITY PACKAGES ***
%
%\usepackage{ifpdf}
% Heiko Oberdiek's ifpdf.sty is very useful if you need conditional
% compilation based on whether the output is pdf or dvi.
% usage:
% \ifpdf
%   % pdf code
% \else
%   % dvi code
% \fi
% The latest version of ifpdf.sty can be obtained from:
% http://www.ctan.org/tex-archive/macros/latex/contrib/oberdiek/
% Also, note that IEEEtran.cls V1.7 and later provides a builtin
% \ifCLASSINFOpdf conditional that works the same way.
% When switching from latex to pdflatex and vice-versa, the compiler may
% have to be run twice to clear warning/error messages.

% *** CITATION PACKAGES ***
%
%\usepackage{cite}
% cite.sty was written by Donald Arseneau
% V1.6 and later of IEEEtran pre-defines the format of the cite.sty package
% \cite{} output to follow that of IEEE. Loading the cite package will
% result in citation numbers being automatically sorted and properly
% "compressed/ranged". e.g., [1], [9], [2], [7], [5], [6] without using
% cite.sty will become [1], [2], [5]--[7], [9] using cite.sty. cite.sty's
% \cite will automatically add leading space, if needed. Use cite.sty's
% noadjust option (cite.sty V3.8 and later) if you want to turn this off.
% cite.sty is already installed on most LaTeX systems. Be sure and use
% version 4.0 (2003-05-27) and later if using hyperref.sty. cite.sty does
% not currently provide for hyperlinked citations.
% The latest version can be obtained at:
% http://www.ctan.org/tex-archive/macros/latex/contrib/cite/
% The documentation is contained in the cite.sty file itself.

% *** GRAPHICS RELATED PACKAGES ***
%
\ifCLASSINFOpdf
  % \usepackage[pdftex]{graphicx}
  % declare the path(s) where your graphic files are
  % \graphicspath{{../pdf/}{../jpeg/}}
  % and their extensions so you won't have to specify these with
  % every instance of \includegraphics
  % \DeclareGraphicsExtensions{.pdf,.jpeg,.png}
\else
  % or other class option (dvipsone, dvipdf, if not using dvips). graphicx
  % will default to the driver specified in the system graphics.cfg if no
  % driver is specified.
  % \usepackage[dvips]{graphicx}
  % declare the path(s) where your graphic files are
  % \graphicspath{{../eps/}}
  % and their extensions so you won't have to specify these with
  % every instance of \includegraphics
  % \DeclareGraphicsExtensions{.eps}
\fi
% graphicx was written by David Carlisle and Sebastian Rahtz. It is
% required if you want graphics, photos, etc. graphicx.sty is already
% installed on most LaTeX systems. The latest version and documentation can
% be obtained at:
% http://www.ctan.org/tex-archive/macros/latex/required/graphics/
% Another good source of documentation is "Using Imported Graphics in
% LaTeX2e" by Keith Reckdahl which can be found as epslatex.ps or
% epslatex.pdf at: http://www.ctan.org/tex-archive/info/
%
% latex, and pdflatex in dvi mode, support graphics in encapsulated
% postscript (.eps) format. pdflatex in pdf mode supports graphics
% in .pdf, .jpeg, .png and .mps (metapost) formats. Users should ensure
% that all non-photo figures use a vector format (.eps, .pdf, .mps) and
% not a bitmapped formats (.jpeg, .png). IEEE frowns on bitmapped formats
% which can result in "jaggedy"/blurry rendering of lines and letters as
% well as large increases in file sizes.
%
% You can find documentation about the pdfTeX application at:
% http://www.tug.org/applications/pdftex

% *** MATH PACKAGES ***
%
%\usepackage[cmex10]{amsmath}
% A popular package from the American Mathematical Society that provides
% many useful and powerful commands for dealing with mathematics. If using
% it, be sure to load this package with the cmex10 option to ensure that
% only type 1 fonts will utilized at all point sizes. Without this option,
% it is possible that some math symbols, particularly those within
% footnotes, will be rendered in bitmap form which will result in a
% document that can not be IEEE Xplore compliant!
%
% Also, note that the amsmath package sets \interdisplaylinepenalty to 10000
% thus preventing page breaks from occurring within multiline equations. Use:
%\interdisplaylinepenalty=2500
% after loading amsmath to restore such page breaks as IEEEtran.cls normally
% does. amsmath.sty is already installed on most LaTeX systems. The latest
% version and documentation can be obtained at:
% http://www.ctan.org/tex-archive/macros/latex/required/amslatex/math/

% *** SPECIALIZED LIST PACKAGES ***
%
%\usepackage{algorithmic}
% algorithmic.sty was written by Peter Williams and Rogerio Brito.
% This package provides an algorithmic environment fo describing algorithms.
% You can use the algorithmic environment in-text or within a figure
% environment to provide for a floating algorithm. Do NOT use the algorithm
% floating environment provided by algorithm.sty (by the same authors) or
% algorithm2e.sty (by Christophe Fiorio) as IEEE does not use dedicated
% algorithm float types and packages that provide these will not provide
% correct IEEE style captions. The latest version and documentation of
% algorithmic.sty can be obtained at:
% http://www.ctan.org/tex-archive/macros/latex/contrib/algorithms/
% There is also a support site at:
% http://algorithms.berlios.de/index.html
% Also of interest may be the (relatively newer and more customizable)
% algorithmicx.sty package by Szasz Janos:
% http://www.ctan.org/tex-archive/macros/latex/contrib/algorithmicx/

% *** ALIGNMENT PACKAGES ***
%
%\usepackage{array}
% Frank Mittelbach's and David Carlisle's array.sty patches and improves
% the standard LaTeX2e array and tabular environments to provide better
% appearance and additional user controls. As the default LaTeX2e table
% generation code is lacking to the point of almost being broken with
% respect to the quality of the end results, all users are strongly
% advised to use an enhanced (at the very least that provided by array.sty)
% set of table tools. array.sty is already installed on most systems. The
% latest version and documentation can be obtained at:
% http://www.ctan.org/tex-archive/macros/latex/required/tools/

%\usepackage{mdwmath}
%\usepackage{mdwtab}
% Also highly recommended is Mark Wooding's extremely powerful MDW tools,
% especially mdwmath.sty and mdwtab.sty which are used to format equations
% and tables, respectively. The MDWtools set is already installed on most
% LaTeX systems. The lastest version and documentation is available at:
% http://www.ctan.org/tex-archive/macros/latex/contrib/mdwtools/

% IEEEtran contains the IEEEeqnarray family of commands that can be used to
% generate multiline equations as well as matrices, tables, etc., of high
% quality.

%\usepackage{eqparbox}
% Also of notable interest is Scott Pakin's eqparbox package for creating
% (automatically sized) equal width boxes - aka "natural width parboxes".
% Available at:
% http://www.ctan.org/tex-archive/macros/latex/contrib/eqparbox/

% *** SUBFIGURE PACKAGES ***
%\usepackage[tight,footnotesize]{subfigure}
% subfigure.sty was written by Steven Douglas Cochran. This package makes it
% easy to put subfigures in your figures. e.g., "Figure 1a and 1b". For IEEE
% work, it is a good idea to load it with the tight package option to reduce
% the amount of white space around the subfigures. subfigure.sty is already
% installed on most LaTeX systems. The latest version and documentation can
% be obtained at:
% http://www.ctan.org/tex-archive/obsolete/macros/latex/contrib/subfigure/
% subfigure.sty has been superceeded by subfig.sty.

%\usepackage[caption=false]{caption}
%\usepackage[font=footnotesize]{subfig}
% subfig.sty, also written by Steven Douglas Cochran, is the modern
% replacement for subfigure.sty. However, subfig.sty requires and
% automatically loads Axel Sommerfeldt's caption.sty which will override
% IEEEtran.cls handling of captions and this will result in nonIEEE style
% figure/table captions. To prevent this problem, be sure and preload
% caption.sty with its "caption=false" package option. This is will preserve
% IEEEtran.cls handing of captions. Version 1.3 (2005/06/28) and later
% (recommended due to many improvements over 1.2) of subfig.sty supports
% the caption=false option directly:
%\usepackage[caption=false,font=footnotesize]{subfig}
%
% The latest version and documentation can be obtained at:
% http://www.ctan.org/tex-archive/macros/latex/contrib/subfig/
% The latest version and documentation of caption.sty can be obtained at:
% http://www.ctan.org/tex-archive/macros/latex/contrib/caption/

% *** FLOAT PACKAGES ***
%
%\usepackage{fixltx2e}
% fixltx2e, the successor to the earlier fix2col.sty, was written by
% Frank Mittelbach and David Carlisle. This package corrects a few problems
% in the LaTeX2e kernel, the most notable of which is that in current
% LaTeX2e releases, the ordering of single and double column floats is not
% guaranteed to be preserved. Thus, an unpatched LaTeX2e can allow a
% single column figure to be placed prior to an earlier double column
% figure. The latest version and documentation can be found at:
% http://www.ctan.org/tex-archive/macros/latex/base/

%\usepackage{stfloats}
% stfloats.sty was written by Sigitas Tolusis. This package gives LaTeX2e
% the ability to do double column floats at the bottom of the page as well
% as the top. (e.g., "\begin{figure*}[!b]" is not normally possible in
% LaTeX2e). It also provides a command:
%\fnbelowfloat
% to enable the placement of footnotes below bottom floats (the standard
% LaTeX2e kernel puts them above bottom floats). This is an invasive package
% which rewrites many portions of the LaTeX2e float routines. It may not work
% with other packages that modify the LaTeX2e float routines. The latest
% version and documentation can be obtained at:
% http://www.ctan.org/tex-archive/macros/latex/contrib/sttools/
% Documentation is contained in the stfloats.sty comments as well as in the
% presfull.pdf file. Do not use the stfloats baselinefloat ability as IEEE
% does not allow \baselineskip to stretch. Authors submitting work to the
% IEEE should note that IEEE rarely uses double column equations and
% that authors should try to avoid such use. Do not be tempted to use the
% cuted.sty or midfloat.sty packages (also by Sigitas Tolusis) as IEEE does
% not format its papers in such ways.

% *** PDF, URL AND HYPERLINK PACKAGES ***
%
%\usepackage{url}
% url.sty was written by Donald Arseneau. It provides better support for
% handling and breaking URLs. url.sty is already installed on most LaTeX
% systems. The latest version can be obtained at:
% http://www.ctan.org/tex-archive/macros/latex/contrib/misc/
% Read the url.sty source comments for usage information. Basically,
% \url{my_url_here}.

% *** Do not adjust lengths that control margins, column widths, etc. ***
% *** Do not use packages that alter fonts (such as pslatex).         ***
% There should be no need to do such things with IEEEtran.cls V1.6 and later.
% (Unless specifically asked to do so by the journal or conference you plan
% to submit to, of course. )

% correct bad hyphenation here
\hyphenation{op-tical net-works semi-conduc-tor}

% -----------------------------------------------------------------------------
% BEGIN DOCUMENT
% -----------------------------------------------------------------------------
\begin{document}

\title{Everything you wanted to know about PanDA but were truly truly afraid to
ask}


% author names and affiliations
% use a multiple column layout for up to two different
% affiliations
\author{\IEEEauthorblockN{Authors Name/s per 1st Affiliation (Author)}
\IEEEauthorblockA{line 1 (of Affiliation): dept.\ name of organization\\
line 2: name of organization, acronyms acceptable\\
line 3: City, Country\\
line 4: Email: name@xyz.com}
\and
\IEEEauthorblockN{Authors Name/s per 2nd Affiliation (Author)}
\IEEEauthorblockA{line 1 (of Affiliation): dept.\ name of organization\\
line 2: name of organization, acronyms acceptable\\
line 3: City, Country\\
line 4: Email: name@xyz.com}
}

% conference papers do not typically use \thanks and this command
% is locked out in conference mode. If really needed, such as for
% the acknowledgment of grants, issue a \IEEEoverridecommandlockouts
% after \documentclass

% for over three affiliations, or if they all won't fit within the width
% of the page, use this alternative format:
%
%\author{\IEEEauthorblockN{Michael Shell\IEEEauthorrefmark{1},
%Homer Simpson\IEEEauthorrefmark{2},
%James Kirk\IEEEauthorrefmark{3},
%Montgomery Scott\IEEEauthorrefmark{3} and
%Eldon Tyrell\IEEEauthorrefmark{4}}
%\IEEEauthorblockA{\IEEEauthorrefmark{1}School of Electrical and Computer Engineering\\
%Georgia Institute of Technology,
%Atlanta, Georgia 30332--0250\\ Email: see http://www.michaelshell.org/contact.html}
%\IEEEauthorblockA{\IEEEauthorrefmark{2}Twentieth Century Fox, Springfield, USA\\
%Email: homer@thesimpsons.com}
%\IEEEauthorblockA{\IEEEauthorrefmark{3}Starfleet Academy, San Francisco, California 96678-2391\\
%Telephone: (800) 555--1212, Fax: (888) 555--1212}
%\IEEEauthorblockA{\IEEEauthorrefmark{4}Tyrell Inc., 123 Replicant Street, Los Angeles, California 90210--4321}}

% use for special paper notices
%\IEEEspecialpapernotice{(Invited Paper)}

% make the title area
\maketitle

% ----------------------------------------------------------------------------
% ABSTRACT
% ----------------------------------------------------------------------------
\begin{abstract}
Experiments at the Large Hadron Collider (LHC) face unprecedented computing
challenges. Heterogeneous resources are distributed worldwide, thousands of
physicists analyzing the data need remote access to hundreds of computing
sites, the volume of processed data is beyond the exabyte scale, and data
processing requires more than billions of hours of computing usage per year. The
PanDA (Production and Distributed Analysis) system was developed to meet the
scale and complexity of LHC distributed computing for the ATLAS experiment. In
the process, the old batch job paradigm of computing in HEP was discarded in
favor of a far more flexible and scalable model. The success of PanDA at the LHC
is leading to widespread adoption and testing by other experiments. PanDA is
the first exascale workload management system in HEP, already operating at
a million computing jobs per day, and processing over an exabyte of data in
2013. We will describe the design and implementation of PanDA, present data on
the performance of PanDA at the LHC, and discuss plans for future evolution
of the system to meet new challenges of scale, heterogeneity and increasing
user base.
\end{abstract}

\begin{IEEEkeywords}
TBD\@.
\end{IEEEkeywords}


% For peer review papers, you can put extra information on the cover
% page as needed:
% \ifCLASSOPTIONpeerreview
% \begin{center} \bfseries EDICS Category: 3-BBND \end{center}
% \fi
%
% For peerreview papers, this IEEEtran command inserts a page break and
% creates the second title. It will be ignored for other modes.
\IEEEpeerreviewmaketitle


% ----------------------------------------------------------------------------
% I - INTRODUCTION
% ----------------------------------------------------------------------------
The Large Hadron Collider (LHC) was created to explore the fundamental
properties of matter for the next decades.  Since LHC start-up in 2009, multiple
experiments  at LHC have collected and distributed hundreds of petabytes of data
worldwide to hundreds of computer centers. Thousands of physicists analyze
petascale data volumes daily. One of the LHC experiments, the
ATLAS~\cite{Aad:2008}, utilizes the Production and Distributed Analysis (PanDA)
workload management system~\cite{Maeno2011} (WMS) for distributed data
processing and analysis. The ATLAS Computing model~\cite{jones2008atlas} is
based on a Grid paradigm~\cite{foster2003grid}, with multilevel, hierarchically
distributed computing and storage resources. PanDA has been developed to meet
growing ATLAS production and analysis requirements for a data-driven workload
management system capable of operating at LHC data processing scale.

% PanDA has a highly scalable architecture. Scalability has been demonstrated in
% ATLAS through the rapid increase in usage over the past several years of
% operations, and PanDA is expected to meet the continuously growing computing
% requirements of ATLAS over the next decade. PanDA was designed to have the
% flexibility to adapt to emerging computing technologies in processing, storage,
% networking as well as the underlying software stack. This flexibility has also
% been successfully demonstrated through the past six years: computing centers in
% ATLAS, spanning many continents, were seamlessly integrated into PanDA\@.

% PanDA manages a wide spectrum of workloads, ranging from raw data processing to
% Monte Carlo simulation and user analysis, while constantly evolving to meet
% rapidly changing science needs.

\mtnote{Used the following in related work. We may want to reduce/remove it from
the introduction: ``Today, PanDA serves several thousand users, managing job
distribution to hundreds of ATLAS sites with more than 100,000 CPU cores which
process more than a million jobs per day.''} % (Figure~\ref{fig:daily}).

% \begin{figure}
%     \begin{center}
%         \includegraphics[width=\columnwidth]{figures/DailyJobs.png}
%         \caption{Daily completed jobs on ATLAS Grid for the past 12 month}
%     \end{center}
% \label{fig:daily}
% \end{figure}

In this paper, we describe how PanDA has been engineered to execute a specific
stage of the ATLAS Monte Carlo workflow on Titan, the larger high-performance
computing HPC system currently available in the USA\@.\mtnote{Explain the
benefits offered by Titan in terms of multithreading per node and possibly large
amount of concurrent nodes. Introduce also the notion of backfill.} This extends
the scope of PanDA's compute model, integrating both high-throughput and
high-performance computing resources and enabling the concurrent execution of
both  single and multi-core jobs. The integration of PanDA and Titan went
through three main engineering phases: (i) feasibility study and rapid
prototyping of an initial solution; (ii) progressing scaling of the  prototype
to saturate the available resources; (iii) study of a product-grade architecture
for generic HPC resources. Both phase i and ii have been completed enabling the
execution of up to eight million jobs a week on Titan. A prototype has been
engineered to support phase iii and experimental data are being collected.

In the next section we introduce \ldots.

Why Titan?
\begin{itemize}
    \item A lot of slow (relative the grid) and homogeneous cores together
    \item Grid is saturated
    \item Enable HPC as a calss of computing resource
    \item Enable future DOE experimental and observational capabilities on HPC
    \item Why is so important: more data, run 2 and run 3.
    \item growth of grid is flat (economic model) and saturated. We need more CPU because we have more data.
    \item ATLAS spend most of the time on simulations this is why we want to offload simulations - that happen happen to be performed via AthenaMP.
\end{itemize}

\mtnote{Moved from related work were we use already this terminology. To be
iterated/adjusted for consistency once the first draft will be ready.}

The term ``workflow'' is used in many disciplines with different meaning. In the
field of scientific computing, ``workflow'' assumes different meanings depending
on the characteristics of the computation, of the software tools used to support
this computation, and of the resources on which it is performed. Further, a
workflow may indicate a whole application, a description of the computational
process of that application or, more commonly, a series of tasks related by data
dependences.

The lack of a consistent and shared definition of ``workflow'' hinders the
understanding of its properties and its relations with related concepts. For
example, we need to clarify the difference among ``workflow'', ``workload'',
``task'', or ``job'' but also between workflow ``template'' and ``instance'', or
``data-flow'' and ``control-flow''. This is precondition to specify properly the
design of software systems that support the execution of scientific
applications.

In this paper, we use the following definitions:

\begin{description}

  \item[Task.] A set of operations to be performed on a computing platform,
  alongside a description of the properties and dependences of those operations,
  and indications on how the dependences should be satisfied and the operations
  should be executed.

  \item[Job.] A unit of  work performed by submitting a script to a resource
  management system (LRMS), like  Slurm or PBS, or by requesting a virtual
  machine or a container to a site supporting virtualization. One or more jobs
  can perform the operations described with a task.

  \item[Workload.] A set of jobs that can be executed concurrently, possibly
  related by a set of relations. For example, jobs of a workload can share one
  or more input files or communicate during execution.

  \item[Workflow.] Set of jobs, related by a set of relations that define the
  order in which each task can be executed. Data dependences are the most common
  relations among workloads, used to define the precedence among their
  executions.

\end{description}


\begin{enumerate}
  \item Problem that has been solved: executing many applications on
  heterogeneous distributed resources.
  \item Why are WMS needed?
  \item Increasing importance of WMS w.r.t complex and scalable applications on
  heterogeneous distributed resources.
  \item \ldots
  \item Overview/Summary of paper.
\end{enumerate}


% ----------------------------------------------------------------------------
% II - BACKGROUND AND RELATED WORK
% ----------------------------------------------------------------------------
\section{Background and Related Work}
\label{sec:bkgrd}

\begin{enumerate}
  \item Brief overview of distributed CI\@.
  \item Workload vs Workflow. (DONE)
  \item Distinguish between WFMS and WLMS\@.
  \begin{enumerate}
    \item Role of Workload MS\@: help ``hide'' and federate heterogeneous resources.
  \end{enumerate}
\end{enumerate}

% Please do not correct bckground to background.
\subsection{Terminology}

\mtnote{To be moved to the end of the introduction.}

The term ``workflow'' is used in many disciplines with different meaning. In the
field of scientific computing, ``workflow'' assumes different meanings depending
on the characteristics of the computation, of the software tools used to support
this computation, and of the resources on which it is performed. Further, a
workflow may indicate a whole application, a description of the computational
process of that application or, more commonly, a series of tasks related by data
dependences.

The lack of a consistent and shared definition of ``workflow'' hinders the
understanding of its properties and its relations with related concepts. For
example, we need to clarify the difference among ``workflow'', ``workload'',
``task'', or ``job'' but also between workflow ``template'' and ``instance'', or
``data-flow'' and ``control-flow''. This is precondition to specify properly the
design of software systems that support the execution of scientific
applications.

In this paper, we use the following definitions:

\begin{description}

  \item[Task.] A set of operations to be performed on a computing platform,
  alongside a description of the properties and dependences of those operations,
  and indications on how the dependences should be satisfied and the operations
  should be executed.

  \item[Job.] A unit of execution that performs one or more unit of work. Jobs
  relates to the resource on which they are executed. One or more tasks can be
  the units of work executed by a job.

  \item[Workload.] A set of tasks that can be executed concurrently, possibly
  related by a set of relations. For example, tasks of a workload can share one
  or more input files or communicate during execution.

  \item[Workflow.] Set of workloads, related by a set of relations that define
  the order in which each workload can be executed. Data dependences are the
  most common relations among workloads, used to define the precedence among
  their executions. Note that, formally, a workload can have a single task.

\end{description}

Each task may have an arbitrary number of properties like number of cores,
executables, or input/output files. Tasks may have precedence interrelations,
depending on their data dependences or any other type of dependency mandated
by the application algorithms. Tasks with precedence relations have to be
executed serially; otherwise, tasks can be executed concurrently. A workload is
defined as a set of tasks that can be executed concurrently. As such, a
workflow can be composed of a set of workloads.

The terms ``task'' and ``job'' are also used inconsistently across communities
that perform scientific computing. In this paper, ``job'' refers to the unit of
work that is submitted to a local resource management system (LRMS), like the
Slurm or PBS batch system of a cluster. As such, a task can become a job when
is scheduled on a resource that exposes a LRMS but can also become a virtual
machine or a container when bootstrapped on an infrastructure supporting
virtualization. Tasks can be statically or dynamically grouped into jobs,
depending on the resource capabilities and the task requirements.

Usually, workflows are represented as graphs in which tasks are vertices and
relations are edges~\cite{}. Often, graphs are supposed to be acyclic but graphs
with cycles have been used to represent workflows of workflows~\cite{}. In this
paper, a workflow template is a type of graph while a workflow instance is a
workflow template with specific vertices and edges. Further, a workflow instance
is ``abstract'' when no resource properties are available for all vertices,
``concrete'' otherwise~\cite{}. When not qualified, the term ``workflow''
indicates a abstract workflow instance.


\begin{enumerate}
  \item Other workload management systems (WMS) (not exclusive)
  \begin{enumerate}
    \item glidein-WMS
    \item Dirac
    \item ALICE-AlieN
  \end{enumerate}
  \begin{enumerate}
    \item Comparison and contrast
    \item Core features of a general WMS, i.e., minimal complete model of a WMS\@.
  \end{enumerate}
\end{enumerate}

\subsection{Alien}
Alien is a Workload and Data Management system composed of a 	set of middleware tools and services entirely based on web-services and standard protocols. Alien was originally developed for the ALICE experiment \cite{} but subsequently used by several virtual organizations \cite{}. 
%The system has been deployed in 2001 for distributed producti	on of Monte Carlo data, detector simulation and reconstruction.
Alien is composed of two type of services:
\begin{itemize}
\item
\emph{Central services}, these services are unique for each virtual organisation, therefore there is only one  configuration point for the management;
\item \emph{Site services}, they provide the interfacing to local resources and Grid services running on a VO-box;
\end{itemize}

The most important central services are:
\begin{itemize}
\item \emph{Job Manager}, a database that keeps track of all the jobs submitted to the system and their current execution status;
\item \emph{Brokers}, they are the core of task executions and data transfers; they receive tasks in form of JDL,  keep them ordered by priority and send them to the CE for execution;
\item \emph{Optimizers}, they are used to minimize the work of the Broker by scanning periodially the task queue and re-arranging the tasks in such a way that fairness and priority policies are guaranted;
\item \emph{Data Catalogue}, it keeps track of the scripts and files uploaded on Storage Elements.
\end{itemize}

%The Computing Agents are instead site services that monitor the local Computing Element, advertise site's capabilities and are responsible for submitting the JobAgents.

%Information about the status of the sites and central services, full job statistics and monitoring information are kept in a MonALISA repository.
%% CLUSTER MONITOR SHOULD BE EQUAL TO COMPUTING AGENT
%% Job Manager should be equal to TaskQueue
%% Process Monitor == PIlot????

The job execution in Alien is usually distributed over several sites. Each of these sites has at least one service called ClusterMonitor. On one side the Cluster Monitor is used to communicate with Central services (Job Manager and Broker), on the other side it can manage Computing Elements (CE) by starting and stopping them whenever it receives the signal.
The CE is the resource in charge of the execution of the jobs. A CE usually is associated with a batch queue and can send the jobs to the worker nodes controlled by the queue.

The CE asks the Broker for jobs to execute by sending its JDL. Once received the JDL, the Broker will try to match it with the JDL of the jobs in queue. If a match exists then the Broker sends the jobs JDL to the CE.
Immediately after receving a job JDL, the CE create a new service on the worker node called ProcessMonitor. This service allows the CE (and the rest of Alien services through the CE) to interact with the job while is running. 
This execution strategy is called ``pull mode'' due to the fact that CE asks for jobs.
It is worth to mention that more recent version of Alien implement exploit gLite and CE-CREAM. The latter allows the system to by-pass the broker during the job submission \cite{}. 

Alien uses the Job Description Language (JDL) to allow users to describe their workload task by task;  i.e.,  users can specify features such as task priorities, the level of parallelism (one core, multi-core, MPI etc..) and  also the DCR that should be targeted for the execution.

%Job submission is implemented by following the so-called ``pull mode'' which is composed of the following steps: 
%\begin{enumerate}
% \item the VO-Box monitors the status of the site queues through polls to the resource running on the CE; 
%\item  the Job Broker receives a report everytime slots become available; 
%\item  if the Task Queue is not empty, the Job Broker asks the VO-box to submit a number of Agents;
%\item finally, the JobAgents are submitted  to the site Computing Element either by way of that sends them back to the site Computing Element or, wherever available, directly through the CREAM interface on the CE itself.
%\end{enumerate}

\subsection{DIRAC} 
DIRAC (Distributed Infrastructure with Remote Agent Control) Workload and Data Management System is a software product, developed within the CERN LHCb project, to manage the processing of detector data, Monte Carlo simulations, and end-user analyses. 
DIRAC  architecture relies on four entities:
\begin{itemize}
\item \emph{Clients}: consist in a set of APIs that allows users to submit job requests. Clients interact directly with DIRAC central services.
\item \emph{Services}: serve Clients and Agents by performing crucial operations such as Job Management, Configuration, Bookkeping and Accounting.
\item \emph{Agents}: perform repetitive tasks like querying file catalogs,  monitoring of jobs on resources.
\item \emph{Resources}: they can be PC's, site cluters and Grids. Agents interact with them without distinction.
\end{itemize}
In the same way of Alien, DIRAC implements a pull scheduling. Furthermore DIRAC was the first WMS to exploit the concept of Pilot Agent on the Grid. 
Pilots Agents are gLite jobs that are submitted to the grid when jobs arrive into the WMS. 
DIRAC pilot system has four main logical components:
\begin{itemize}
\item a set of TaskQueues that collect tasks submitted by users, multiple TaskQueue being created depending on the requirements and ownership of the tasks;
\item a set of JobWrappers that are executed on the DCR to bind compute resources and execute tasks submitted by the users;
\item a set of TaskQueueDirectors that submits JobWrappers to target DCRs;
\item a MatchMaker that matches requests from JobWrappers to suitable tasks into TaskQeues.
\end{itemize}
The DIRAC execution model can be summarized in five
steps: 1. a user submits its workload in form of tasks to the WMS Job Manager; 2. submitted tasks are validated and added to a new or an existing TaskQueue, depending on the task properties; 3. TaskQueueDirector evaluates TaskQueues and a suitable number of JobWrappers are submitted to available
DCRs; 4. JobWrappers get instantiated on the DCRs and, then,  ask for tasks to the MatchMaker; 5. JobWrappers execute tasks while JobWrapper’s Watchdog monitor them.
TaskQueueDirectors deploy Pilots by getting a list of TaskQueues and calculating the number of pilot to submit  according to user priorities.
Once deployed on the compute resource, Pilots, a.k.a. JobWrappers, hold the resource in the form of single or multiple cores, spanning portions, whole, or multiple compute nodes. Pilots do not expose data capabilities although the system allows the user to perform both data staging and data replication. 
TaskQueues, TaskQueueDirectors, and the MatchMaker are implemented as services whereas the JobWrapper is implemented within the Agents together with the WatchDog. 

\subsection{HTCondor Glidein and GlideinWMS}
The HTCondor Glidein system  as part of the HTCondor software ecosystem. The HTCondor Glidein is a pilot based system to aggregate DCRs with heterogeneous middleware into HTCondor resource pools.
Condor is based on daemons collaborating by exchanging messages over the network. We can isolate four main logical components:
\begin{itemize}
\item \emph{Schedd}, implements a queuing system that holds workload tasks;
\item \emph{Startd}, controls the DCR resources. 
\item \emph{Collector}, holds references to all the active
Schedd/Startd daemons; 
\item \emph{Negotiator} matches tasks queued in a Schedd to resources handled by a Startd.
\end{itemize}
Glidein-WMS has been developed to integrate HTCondor Glidein to  automate the deployment and management of Glideins on multiple types of DCR middleware. 
The integration required three additional logical components: 
\begin{itemize}
\item \emph{Glidein Factories} that submit tasks to the DCRs middleware;
\item a set of \emph{Virtual Organizations (VO) Frontend} daemons that match the tasks on one or more Schedd to the resource attributes;
\item a \emph{Collector} that holds references to all the active Glidein Factories and VO Frontend daemons. 
\end{itemize}

 The execution model of the HTCondor Glidein system can be summarized in nine steps: 1. the user submits a Glidein (i.e., a job) to a DCR batch scheduler; 2. once executed, this Glidein bootstraps a Startd daemon; 3. the Startd daemon advertises itself with the Collector; 4. the user submits the tasks of the workload to the Schedd daemon; 5. the Schedd advertises these tasks to the Collector; 6. the Negotiator matches the requirements of the tasks to the properties of one of the available Startd daemon (i.e., a Glidein); 7. the Negotiator communicates the match to the Schedd; 8. the Schedd submits the tasks to the Startd daemon indicated by
the Negotiator; 9. the task is executed.

By using GlideinWMS, the user does not have to submit Glidein directly but only tasks to Schedd. From there: 1. every Schedd advertises its tasks with the VO Frontend; 2. the VO Frontend matches the tasks’ requirements to the resource properties advertised by the WMS Connector; 3. the VO Frontend places requests for Glideins instantiation to the WMS Collector; 4. the WMS Collector contacts the appropriate Glidein Factory to execute the requested Glideins; 5. the requested Glideins become active on the DCRs; and 6. the Glideins advertise their availability to the (HTCondor) Collector. From there on the execution model is the same as described for the HTCondor Glidein Service.

The resources managed by a single Glidein (i.e., pilot) are limited to compute resources. Glideins may bind one or more cores, depending on the target DCRs. For example, heterogeneous HTCondor pools with resources for desktops, workstations, small campus clusters, and some larger clusters will run mostly single core Glideins. More specialized pools that hold, for example, only DCRs with HTC, Grid, or Cloud middleware may instantiate Glideins with a larger number of cores. Both HTCondor Glidein and GlideinWMS provide abstractions for file staging but pilots are not used to hold data or network resources.
The process of pilot deployment is the main difference between HTCondor Glidein and GlideinWMS. While the
HTCondor Glidein system requires users to submit the pilots to the DCRs, GlideinWMS automates and optimizes pilot provisioning. GlideinWMS attempts to maximize the throughput of task execution by continuously instantiating Glideins until the queues of the available Schedd are emptied. Once all the tasks have been executed, the remaining Glideins are terminated.
HTCondor Glidein and GlideWMS expose the interfaces of HTCondor to the application layer and no theoretical
limitations are posed on the type and complexity of the workloads that can be executed. For example, DAGMan
(Directed Acyclic Graph Manager) has been designed to execute workflows by submitting tasks to Schedd, and a tool is available to design applications based on the master-worker coordination pattern.

Both HTCondor Glidein and GlideWMS rely on one or more HTCondor Collectors to match task requirements and resource properties, represented as ClassAds. This matching can be evaluated right before the scheduling of the task. In this way, late binding is achieved but early binding remains unsupported.
\begin{table*}
\begin{center}
\begin{tabular}{llllllll}
  \hline
Pilot System  &Logical Components& Execution Strategy & Binding  & Workload Definition  &  Broker  & \\
\hline
Alien & Central services, site services & pull, pilot via Co-pilot & Late & JDL & Condor (ClassAd attributes) &\\
DIRAC & Services, Agents& pull, pilot-based & Late & JDL, WF (TMS) & Condor(ClassAd attributes) &\\
glidein WMS& Daemons & pull, pilot-based  & Late & Pegaus, DAGMan & Condor(ClassAd attributes) &\\
\hline
\end{tabular}
\end{center}
\caption{Comparison of the three Workload and Data Management Systems}\label{tab:Summary}
\end{table*}
	



% ----------------------------------------------------------------------------
% III - PanDA OVERVIEW
% ----------------------------------------------------------------------------
\section{PanDA Overview}
\label{sec:panda_overview}

PanDA consists of several interconnected subsystems, most of them are built from off-the-shelf and Open Source components. Figure \ref{fig:architecture} shows a schematic view of the PanDA system.
In the following we will briefly describe system’s architecture and components.

\begin{itemize}
\item PanDA Server
\item Pilot
\item Factory
\item PanDA monitoring
\item JEDI ??
\item What else?
\end{itemize}

\begin{figure}
\begin{center}
\includegraphics[width=\columnwidth]{figures/PandaArch.jpg}
\caption{Schematic view of the PanDA system\label{fig:architecture}. 
  Originally PanDA was designed for grid infrastructure... In this paper
  is focussed on the HPC use case; there are differences from  the grid use
  case. This paper discusses how PanDA has been adapted to execute a workload on
  HPC resources. The schematic overview is presented for workload X on Titan......}
\end{center}
\end{figure}
\subsection{PanDA Server}
The PanDA server is the main component of the system, It provides  a task queue managing all job information centrally. The PanDA  server receives jobs through the client interface into the task queue, upon which a brokerage module  operates to prioritize  and assign work on the basis of job type, priority, input data and its locality, and available CPU resources. The PanDA  server operates as a web service. It runs on Apache web server, interacting with the back-end database running  on separate servers. It uses the Apache worker  model, with many independent   processes  handling client requests  in  parallel. Since all  state is maintained  in  the central database,  the PanDA Server application instance itself is stateless. Currently production version of PanDA utilizes Oracle database backend, but PanDA  can also work with MySQL family of databases.

\subsection{PanDA Pilot}
 PanDA is a pilot based workload  management system. In the PanDA job lifecycle, pilot jobs (Python scripts that organize workload processing on a worker  node) are submitted to sites. When these  pilot jobs start on a  worker node they contact a central  server  to retrieve  a real payload (i.e., an end-user job) and execute it. Using these pilot-based  workflows helps to improve job reliability, optimize resource utilization, allows for opportunistic  resources usage, and mitigates  many of the problems  associated with the inhomogeneities found on the Grid.
The pilots themselves  do not contain all the functionality needed to request a payload  job from the Panda Server. First, the complete  set of pilot code is downloaded via HTTP from a  central Subversion repository.  The repository works with Apache web server configured  with a  memory-based   web proxy (Squid). The purpose  of the cache  is to reduce  the request load on the back-end Subversion  server. That allows for a  very good performance since only occasional  queries trigger a  full  lookup on the back-end  Subversion  system, and most external  queries are pulled from memory on the front-end web server. The benefits of this system include a high-performance code download service combined with code updates still being immediately  available  as soon as they are committed to source code control.

\subsection{PanDA Factory}
As a pilot-based  system, PanDA  requires some way to get the initial PanDA pilot onto worker nodes at sites. This is done with the help of a component  called AutoPyFactory (APF). APF runs in a single daemonized process, launching  a separate thread for each internal  workflow. Each one of these internal workflows typically serves  a single job queue  as defined  in WMS, and delivers pilots to a single batch queue, either local or remote. The behavior of these APF workflows is determined by the combination of a set  of plugins, invoked in a  fixed order, in a loop, each one in charge of the performance of a well defined action.

\subsection{PanDA Monitoring}
The PanDA Monitor is a web-based dashboard-style  graphical application. It runs on Apache and interacts with the back-end database for persistence. The Monitor is the primary way that users and site administrators get a  view into the status of current and past jobs, data movement,  pilot factory. The Monitor allows users  access  to all job log files on job by job basis thus greatly simplifying code debugging and failure analysis in distributed computing environment.









% ----------------------------------------------------------------------------
% IV - DEPLOYING PANDA ON A LEADERSHIP-SCALE SYSTEM
% ----------------------------------------------------------------------------
\section{Deploying PanDA on a Leadership-scale system}
\label{sec:panda_deployment}

\begin{enumerate}
  \item Why? To support a specific stage of the ATLAS Monte Carlo Workflow.
  \item Challenges
  \item Opportunities
  \item Highlight distinction of PanDA on wide-area distributed systems vs on TITAN
  \item Take-aways, lessons learned
\end{enumerate}

\subsection{OLCF and Titan}

In 2004, the Oak Ridge Leadership Computing Facility (OLCF) was established at
Oak Ridge National Laboratory (ORNL). The OLCF's mission is to accelerate
scientific discovery and engineering progress by providing outstanding computing
and data management resources to high-priority research and development
projects. Among other resources, OLCF manages Titan, a leadership-class and the
fastest open-science supercomputer in the US. Titan enables scientists to
evaluate and assess various complex physical phenomena via large-scale
computational simulations.

Titan is a Cray XK7 system with 18,688 compute nodes~\cite{top500}. Each compute
node is equipped with an AMD Opteron CPU and a Nvidia Tesla GPU. This hybrid
design provides improved energy efficiency, as well as an order of magnitude in
computational capacity over its predecessor. Titan has 710 TB of total system
memory and a center-wide parallel file system known as Spider II~\cite{spider2}.
Spider II is one of the world's fastest and largest POSIX complaint parallel
file systems, designed to serve write-heavy I/O workloads generated by Titan
compute clients and other OLCF resources.

The upcoming LHC Run 3 will require more resources than the Worldwide LHC
Computing Grid (WLCG) can provide. Currently, PanDA WMS uses more than 300,000
cores at over 100 Grid sites, with a peak performance of 0.3 petaFLOPS. This
capacity will be sufficient for the planned analysis and data processing, but it
will be insufficient for the Monte Carlo production workflow and any extra
activity. To alleviate these challenges, ATLAS is engaged in a program to expand
the current computing model to include additional resources such as the
opportunistic use of supercomputers.

Generally, supercomputers are designed to support parallel computation that
requires runtime communication. Jobs are executed across multiple cores, each
core calculating a small part of the problem and communicating with other cores
via MPI. Accordingly, supercomputers have large number of worker nodes,
connected through a high-speed, low-latency dedicated network. Each worker node
has multicore CPUs, usually augmented with Graphics Processing Units (GPUs) or
other specialized coprocessors.

PanDA WMS has been designed to support distributed Grid computing. Executing
ATLAS workloads or workflows involves concurrent and/or sequential runs of
possibly large amount of jobs, each requiring no or minimal parallelization and
no runtime communication. Thus, computing infrastructure like WLCG have been
designed to aggregate large amount of computing resources across multiple sites.
While each site may deploy MPI capabilities, usually these are not used to
perform distributed computations.

We developed and deployed a single-point solution to better understand the
problem space of enabling a WMS designed for HTC to execute production workflows
on resources designed to support HPC. The PanDA team developed a job broker to
support the execution of part of the ATLAS production Monte Carlo workflow on
Titan, a leadership-class supercomputer managed by the Oak Ridge Leadership
Computing Facility (OLCF) at the Oak Ridge National Laboratory (ORNL).


% -----------------------------------------------------------------------------
\subsection{Architectures and Interfaces}
\label{ssec:panda-titan}

The Titan supercomputer, current number three on the Top 500 list~\cite{top500},
is a Cray XK7 system with 18,688 worker nodes and a total of 299,008 CPU cores.
Each worker node has an AMD Opteron  6274 16-core CPU, a Nvidia Tesla K20X GPU,
32 GB of RAM and no local storage, though a 16 GB RAM disk can be set up. Work
nodes use Cray’s Gemini interconnect for inter-node MPI messaging. Titan is
served by the Spider II~\cite{oral2013olcf}, a Lustre filesystem with 32 PB of
disk storage, and by a 29 PB HPSS tape storage system. Titan’s worker nodes run
Compute Node Linux, a run time environment based on SUSE Linux Enterprise
Server.

Titan's users submit jobs to Titan's PBS scheduler by logging into login or data
transfer nodes (DTNs). Titan's authentication and authorization model is based
on two-factor authentication with a RSA SecurID key. Login nodes and DTNs have
out/inbound wide area network connectivity while worker nodes have only local
network access. Fair-share and allocation policies are in place both for the PBS
batch system and shared file systems.

Titan's architecture, configuration and policies poses several challenges to the
integration with PanDA. The default deployment
model of PanDA Pilot is unfeasible on Titan: PanDA Pilot is required to contact
the Job Dispatcher of the PanDA Server to pull jobs to execute, but this is not
possible on Titan because worker nodes do not offer outbound network
connectivity. Further, Titan does not support PanDA's security model based on
certificates and virtual organizations, making PanDA's approach to identity
management also unfeasible. While DTNs offer wide area network data transfer, an
integration with ATLAS DDM is beyond the functional and administrative scope of
the current prototyping phase. Finally, the specific characteristics of the
execution environment, especially the absence of local storage on the worker
nodes and modules tailored to Compute Node Linux, require re-engineering of
ATLAS application frameworks.

Currently, very few HEP applications can benefit from Titan's GPUs but some
computationally-intensive and non memory-intensive tasks of ATLAS workflows can
be off-loaded from the Grid to Titan's. Further, when HEP tasks can be
partitioned into independent jobs, Titan worker nodes can be used to execute up
to 16 concurrent payloads, one per each available core. Given these constraints
and challenges, the type of task most suitable for execution at the moment on
Titan is Monte Carlo detector simulation. This type of task is mostly
computational-intensive, requiring less than 2GB of RAM at runtime and with
small input data requirements. Detector simulation tasks in ATLAS are performed
via AthenaMP~\cite{aad2010atlas}, the ATLAS software framework integrating the
GEANT4 simulation toolkit~\cite{agostinelli2003geant4}. These tasks account for
$\approx$ 60\% of all the jobs on WLCG, making them a primary candidate for
offloading.

Detector simulation is part of the ATLAS production Monte Carlo (MC)
workflow~\cite{rimoldi2006atlas,de2013delphes,ritsch2014atlas}. The MC workflow
consists of four main stages: event generation, detector simulation,
digitization, and reconstruction. Event generation creates sets of particle
four-momenta via different generators, e.g., PYTHIA~\cite{sjostrand2006pythia},
HERWIG~\cite{corcella2001herwig} and many others. The detector simulator is
called Geant4~\cite{agostinelli2003geant4} and simulates the ATLAS detector and
the interaction between that detector and particles. Each interaction creates a
so-called hit and all hits are collected and passed on for digitalization, where
hits are further processed to mimic the readout of the detector. Finally,
reconstruction operates local pattern recognition, creating high-level objects
like particles and jets.

% -----------------------------------------------------------------------------
\subsection{PanDA Broker}
\label{ssec:panda_titan}

The lack of wide area network connectivity on Titan's worker nodes is the most
relevant challenge for integrating PanDA WMS and Titan. Without connectivity,
Panda Pilots cannot be scheduled on worker nodes because they would not be able
to communicate with PanDA Server and therefore pull and execute jobs. This makes
impossible to port PanDA Pilot to Titan while maintaining the defining feature
of the pilot abstraction: decoupling resource acquisition from workload
execution via multi-stage scheduling.

The unavailability of pilots is a potential drawback when executing distributed
workloads like MC detector simulation. Pilots are used to increase the
throughput of distributed workloads: while pilots have to wait in the
supercomputer's queue, once scheduled, they can pull and execute jobs
independent from the system's queue. Jobs can be concurrently executed on
every core available to the pilot, and multiple generations of concurrent
executions can be performed until the pilot's walltime is exhausted. This is
particularly relevant for machines like Titan where queue policies privilege
parallel jobs on the base of the number of worker nodes they request: the higher
the number of nodes, the shorter the amount of queue time (modulo fair-share and
allocation policies).

The backfill optimization of Titan's Moab scheduler allows to avoid the overhead
of queue wait times without using pilot abstraction~\cite{maui_backfill_url}.
With this optimization, Moab starts low-priority jobs when they do not delay
higher priority jobs, independent of whether the low-priority jobs were queued
after the high-priority jobs.

When the backfill optimization is enabled, users can interrogate Moab about the
number of worker nodes and walltime that would be available to a low-priority
job at that moment in time. If a job is immediately submitted to Titan with that
number of worker nodes and walltime, chances are that Moab will immediately
schedule it, reducing its queue time to a minimum. In this paper, we call this
number of worker nodes and walltime an available `backfill slot'.

Compared to pilots, backfill has the disadvantage of limiting the amount of
worker nodes that can be requested. Pilots are normal jobs: they can request as
many worker nodes and walltime as a queue can offer. On the contrary, jobs sized
according to an available backfill slot depend on the number of worker nodes and
walltime that cannot be given to any other job at that moment in time.

At any point in time, the size of an available backfill slot is typically a
small fraction of the total capacity of a resource. Notwithstanding, given the
size of Titan this translates into a substantial capacity. Every year, about
10\% of Titan's capacity remains unused~\cite{barker2016us}, corresponding to an
average of 30,000 unused cores (excluding GPU cores). This equals to roughly
10\% of the overall capacity of WLCG.

Given the communication requirements of PanDA Pilots and the unused capacity of
Titan, PanDA pilot was repurposed to serve as a job broker on the DTN nodes of
Titan (Fig.~\ref{fig:panda_broker}). Maintaining the core modules of PanDA Pilot
and its stand-alone architecture, this prototype called `PanDA Broker'
implements functionalities to: (i) interrogate Titan about backfill
availability; (ii) pull ATLAS jobs and events from PanDA Server; (iii) wrap the
payload of ATLAS jobs into MPI scripts; (iv) submitting MPI scripts to Titan's
PBS batch system and monitor their execution; and (v) staging and preparing
input/output files. Backfill querying, payload wrapping, and scripts submission
required a new implementation while pulling ATLAS job and events, and file
staging were inherited from PanDA Pilot.

Backfill querying is performed via a dedicated Moab scheduler command while a
tailored Python MPI script is used to execute the payload of ATLAS jobs. This
MPI script enables the execution of unmodified Grid-centric, ATLAS jobs on
Titan. Typically, a MPI script is workload-specific as it sets up the execution
environment for a specific payload. This involves organization of worker
directories, data management, optional input parameters modification, and
cleanup on exit. Upon submission, a copy of the MPI script runs on every
available worker node, starting the execution of the ATLAS job's payload in a
subprocess and waits until its completion.

MPI scripts are submitted to Titan's PBS batch system via
RADICAL-SAGA~\cite{radical-saga_url}, a Python module, compliant with the OGF
GFD.90 SAGA specification~\cite{goodale2008simple}. The Simple API for Grid
Applications (SAGA) offers a unified interface to diverse job schedulers and
file transferring services. In this way, SAGA provides an interoperability layer
that lowers the complexity of using distributed infrastructures. Behind the API
façade, RADICAL-SAGA implements a adaptor architecture: each adaptor interface
the SAGA API with different middleware systems and services, including the PBS
batch scheduler of Titan.

The data staging capabilities of the PanDA Broker are implemented via a file
system that is shared among DTNs and worker nodes. The input files with the
events of the ATLAS jobs are downloaded on the shared filesystem from the data
center of Brookhaven National Laboratory (BNL). The MPI script setup process
includes making the location of these files available to the payload of the
ATLAS's jobs. The PanDA Broker can locate the payload's output files on the
shared filesystem and transfer them from Titan BNL.

Once deployed on Titan, every PanDA Broker supports the execution of MC detector
simulations in 9 steps. PanDA Broker queries the Job Dispatcher module of the
PanDA server for ATLAS jobs that have been bound to Titan by JEDI
(Fig.~\ref{fig:panda_broker}:1). Upon receiving jobs descriptions, PanDA Broker
pulls jobs' input files from BNL to the OLCF Lustre file system
(Fig.~\ref{fig:panda_broker}:2). PanDA Broker queries Titan's Moab scheduler
about the current available backfill slot (Fig.~\ref{fig:panda_broker}:3) and
creates an MPI script, wrapping enough ATLAS jobs' payload to fit the backfill
slot. PanDA Broker submits the MPI script to the Titan's Torque batch system via
RADICAL-SAGA (Fig.~\ref{fig:panda_broker}:4).

Upon execution on the worker node(s) (Fig.~\ref{fig:panda_broker}:5), the MPI
script initializes and configures the execution environment
(Fig.~\ref{fig:panda_broker}:6), and executes one AthenaMP for each available
work node (Fig.~\ref{fig:panda_broker}:7). AthenaMP retrieves events from Lustre
(Fig.~\ref{fig:panda_broker}:8) and spawns 1 Geant4 event simulation process on
each of the 16 available cores (Fig.~\ref{fig:panda_broker}:9). Upon completion
of each MPI script, PanDA Broker transfer the jobs' output to BNL
(Fig.~\ref{fig:panda_broker}:10), and performs cleanup.

While PanDA Broker implementation is resource specific, it was successfully
ported to other supercomputers, including the HPC2 at the National Research
Center ``Kurchatov Institute'' (NRC-KI)~\cite{belyaev2015integration},
Edison/Cori at the National Energy Research Scientific Computing Center
(NERSC)~\cite{barreiro2016panda}, and SuperMUC at the Leibniz Supercomputing
Centre (LRZ)~\cite{barreiro2016panda}.

\begin{figure}
    \centering
    \includegraphics[width=\columnwidth]{figures/panda_broker_architecture.pdf}
    \vspace{-0.3in}
    \caption{PanDA Broker architecture as deployed on Titan. Numbers indicates
    the execution process of a detector simulation job, part of the production
    ATLAS MC workflow.}
\label{fig:panda_broker}
\end{figure}



% ----------------------------------------------------------------------------
% VI - PERFORMANCE CHARACTERIZATION
% ----------------------------------------------------------------------------
\section{Performance Characterization}
\label{sec:panda_titan}

% -----------------------------------------------------------------------------
% \subsection{Geant4 Tasks Performance Analysis on Titan}
% \label{ssec:panda_titan}

Currently, 20 instances of PanDA Broker are deployed on a total of 4 DTNs, 5
instances for each DTN. Each broker controls from 15 to 300 detector simulation
jobs per submission, for a theoretical maximum concurrent use of 96,000
cores. Since November 2015, the PanDA Brokers have operated only in backfill
mode, without a defined time allocation, and running at the lowest priority on
Titan.

Figure~\ref{fig:core-hours-utilization} shows Titan core hours used by ATLAS
from January 2016 to February 2017. During that period, ATLAS consumed a total
of 73.8 million Titan core hours, for an average of 7M hours a month, with a
minimum of 3.3M hours in April 2016 and a maximum 14.8M hours in February 2017.
On February, PanDA Brokers used almost twice as much backfill availability than
in any other month. This is likely due to the hardware upgrades made to the
DTNs. The absence of continuous monitoring of the DTNs does not allow for the
specific identification of the bottleneck but spot measurements of the DTNs load
indicate that a faster CPU and better networking were likely responsible for the
improved performance. No relevant code update was made between January and
February 2017 and PanDA Brokers logs indicated that they were able to respond
more promptly to backfill availability.

\begin{figure}[htp]
    \subfloat[]{
        \includegraphics[clip,width=\columnwidth]{figures/cpu_hours.png}
        \label{fig:core-hours-utilization}
    }

    \subfloat[]{
        \includegraphics[clip,width=\columnwidth]{figures/backfill_consumption.png}
        \label{fig:backfill-utilization}
    }
\caption{(a).(b).}
\end{figure}

Investigations of the CPU load of the upgraded DTNs shows an average
XX\mtnote{ask Danila for data} load. This indicates that further hardware
upgrades would not be likely to improve significantly the performance of the
PanDA Brokers. Nonetheless, the current load suggests that the number of brokers
per DTN could be increased. This would enable the submission of a larger number
of concurrent jobs to the Titan's PBS queue, allowing for PanDA to consume a
higher percentage of the overall backfill availability.

Figure~\ref{fig:backfill-utilization} shows backfill utilization efficiency,
defined as the fraction of Titan’s total cores available via backfill utilized
by ATLAS, from January 2016 to February 2017. ATLAS reached 18\% average
efficiency with a minimum 8.9\% efficiency on April 2016 and a maximum 33.5\%
efficiency on February 2017. The number of total backfill cores available in
April 2016 was 38.1M and in February 2017 33.1M. This shows that the improvement
in the efficiency was determine by the different in total backfill availability.

Figure~\ref{fig:hpc-workload-utilization} shows that between January 2016 and
February 2017, about XX million detector simulation jobs were completed on
Titan, for a total of XXXM events processed. This is equivalent to XX\% of the
total amount of detector simulations performed by ATLAS in the same period of
time, and XX\% of the total number of events processed. Comparatively, Titan
contributed between XX\% and XX\% of the total WLCG availability. When
accounting for the amount of unused backfill availability and the rate of
improvement of PanDA efficiency, these figures confirms the fundamental role
that supercomputers' resources can play for the LHC Run 3.

\begin{figure}[htp]
    \includegraphics[clip,width=\columnwidth]{figures/cpu_hours.png}
\caption{\mtnote{Placeholder for the diagram/data I asked to Danila and
Sergey}}
\label{fig:hpc-workload-utilization}
\end{figure}

The scalability of PanDA Broker depends on the DTNs' resources. Each PanDA
Broker can execute 1 job on up to 300 work nodes for a total of 4,800 cores.
Scaling above this threshold requires instantiating multiple PanDA Brokers on
the same DTN and, once the DTN resources are saturated, on multiple DTNs. In
principle, this is not a problem on Titan as DTNs are supposed to be available
and can be periodically upgraded. Assuming the efficiency reached on February as
reference, using all the available backfill resources on Titan will require to
instantiate around 60 PanDA Brokers. It is not clear whether this is within the
scope of the current DTN capabilities, especially when considering the increased
staging in and out of files from the DTNs and ATLAS Grid sites.

The current design and architecture of the PanDA Broker proven to be very
robust. The failure rate of jobs produced by issues with PanDA Broker and PanDA
in general is XX\%. This confirms the benefits given by reusing most of the code
base of the PanDA Pilot and adopting RADICAL-SAGA for communicating with Titan's
PBS batch system and monitoring jobs execution. During XX months of executing
production jobs, most of the failures were due to: \ldots, \ldots, and \ldots.
The usual corrective measure were put in place and the progressive decline of
the failure rate confirms that PanDA Broker reached production-grade stability
on Titan.\mtnote{I requested figures about failure types and rate to Sergey and
Danila.}

\begin{itemize}
    \item Comparison hardware performance Titan/Grid
    \item problem with lustre addressed with ramdisk
    \item Summary overall figure of performance analogous to those produced with NGE. Diagram as discussed in F2F meeting at Rutgers.
    \item Competition for cache in AMD processors.
\end{itemize}

Currently, two main parameters measure the performance of the detector
simulation jobs submitted to Titan: (i) the time taken to setup
AthenaMP~\cite{aad2010atlas}, the ATLAS software framework integrating the
GEANT4 simulation toolkit; and (ii) the distribution of the time taken by the
Geant4 toolkit to simulate a certain number of events.

AthenaMP has an initialization and configuration stage. At initialization time,
AthenaMP is  assembled from a large number of shared libraries. Once
bootstrapped, every algorithm and service is configured by a set of Python
scripts. Both these operations result in a large number of read operations,
including those required to  access of small python scripts.


% AthenaMP is a multipurpose framework that needs to be configured depending on
% the type of payalod that needs to be executed. For Geant4, this configuration
% process links up to 200 libraries, all requiring filesystem read and write
% operations.

% The execution of Geant4 is mostly compute-intensive requiring to write around
% XXKB per 100 events on disk and no more than 2GB of memory.

% -----------------------------------------------------------------------------
% \subsection{PanDA Shared Library I/O Performance Impact at OLCF}

% Athena, the ATLAS framework has a configuration and initialization stage. At
% this stage, the running job is assembled on the fly from a large number of
% shared libraries. Also, at this stage, every algorithm and service is being
% configured, at run time, by a corresponding set of Python scripts, which
% results in a large number of read operations accesses to small python
% scripts, with many includes and imports Python calls.

Originally, all shared libraries of AthenaMP and the python scripts for job
configuration were stored on the Spider 2 Lustre file system. However, the I/O
patterns of the initialization and configuration stages degraded the performance
of the filesystem. Since Spider 2 is a center-wide file system, shared by all
OLCF resources and users, this resulted in lower overall interactive and
metadata performance at OLCF\@. \mtnote{I am afraid the details about the trace
are too specific given the space constraints of a SC submission. Please feel
free to uncomment it if you disagree.}
% As can be seen in Listing~\ref{mdstrace} the metadata I/O activity for ATLAS
% exhibits a spike corresponding to the beginning of the runs before tapering
% off.

% \begin{minipage}{\linewidth}
% \begin{lstlisting}[language=bash,frame=single,basicstyle=\ttfamily\tiny,caption=ATLAS metadata trace,label=mdstrace]
% XK7 Application 9205593
%       39012 RPCs from 300 of 300 nodes
%         ~69291.96 per sec
%           37851 LDLM_ENQUEUE RPCs    ~67229.83 per sec
%                 pmin 13us pavg 42us pmax 4983us
%           611 LDLM_CANCEL RPCs    ~1085.24 per sec
%                 pmin 10us pavg 16us pmax 32us
%           277 MDS_CLOSE RPCs    ~492.00 per sec
%                 pmin 15us pavg 20us pmax 38us
%           154 MDS_READPAGE RPCs    ~273.53 per sec
%                 pmin 170us pavg 292us pmax 671us
%           86 MDS_GETXATTR RPCs    ~152.75 per sec
%                 pmin 15us pavg 21us pmax 65us
%           30 MDS_GETATTR RPCs    ~53.29 per sec
%                 pmin 16us pavg 20us pmax 27us
%           3 MDS_REINT RPCs    ~5.33 per sec
%                 pmin 103us pavg 136us pmax 196us
%           Overall times
%                 pmin 10us pavg 42us pmax 4983us
%
% XK7 Application 9205355
%       8698 RPCs from 62 of 62 nodes
%         ~15449.13 per sec
%           8445 LDLM_ENQUEUE RPCs    ~14999.76 per sec
%                 pmin 16us pavg 41us pmax 780us
%           92 MDS_CLOSE RPCs    ~163.41 per sec
%                 pmin 15us pavg 23us pmax 52us
%           55 MDS_READPAGE RPCs    ~97.69 per sec
%                 pmin 189us pavg 291us pmax 534us
%           52 LDLM_CANCEL RPCs    ~92.36 per sec
%                 pmin 12us pavg 16us pmax 25us
%           41 MDS_GETXATTR RPCs    ~72.82 per sec
%                 pmin 16us pavg 20us pmax 27us
%           13 MDS_GETATTR RPCs    ~23.09 per sec
%                 pmin 16us pavg 21us pmax 38us
%           Overall times
%                 pmin 12us pavg 42us pmax 780us
%
% \end{lstlisting}
% \end{minipage}

% As this problem was identified, the OLCF staff has enabled read-only access to
% certain NFS-exported directories from Titan compute nodes. This in turn
% allowed the OLCF staff to install a software package from a Titan login node
% and have it available read-only on a Titan compute node.

This issue was addressed by moving the AthenaMP distribution to a read-only NFS
directory made accessible from the Titan work nodes. This eliminated the
metadata contention problem improving metadata read performance of two orders of
magnitude: from \~6,300 seconds on Lustre to \~1,500 seconds on NFS. Further, at
configuration time, the input files of each job were stored to a ramdisk on the
work nodes. This offered three orders of magnitude performance improvement in
the file validation step: from 1,320 seconds on Lustre to 40 seconds on NFS.

%Figure~\ref{fig:atlas-perf-improvement} shows the overall ATLAS performance
%improvement on Titan. The circled region illustrates the switch from Lustre to
%NFS-exported directory for hosting the ATLAS release.

%\begin{figure}[!htb]
%    \centering
%    \begin{tabular}{cc}
%        {\includegraphics[width=0.48\textwidth]{figures/panda-completed-jobs-sw-move.pdf}}\\
%    \end{tabular}
%    \caption{ATLAS performance improvement on Titan. The circled region shows the switch from Lustre to NFS-exported directory for hosting the ATLAS release.}
%\label{fig:atlas-perf-improvement}
%\end{figure}


% -----------------------------------------------------------------------------
\subsection{PanDA I/O Impact at OLCF}

To better understand the I/O impact of ATLAS PanDA project on Titan
supercomputing environment we analyzed 1,175 jobs ran on the week of 10/25/2016,
for a total of 174 hours. Table~\ref{panda-olcf-stats} shows the overall
statistical breakdown of the observed file I/O impact of ATLAS at OLCF\@.
Figures~\ref{fig:atlas-titan-io-read} and~\ref{fig:atlas-titan-io-written} show
the file read and write I/O histograms for these 1,175 jobs, respectively.
Figures~\ref{fig:atlas-titan-file-open} and~\ref{fig:atlas-titan-file-close}
show the file $open()$ and $close()$ metadata load histograms of the same 1,175
ATLAS jobs, respectively.

As can be seen from Table~\ref{panda-olcf-stats}, the number of nodes used by
ATLAS jobs vary between 1 and 300, while the average is at 35. 75\% of the ATLAS
jobs consume less than 25 and 92\% consume less than or equal to 100 Titan
compute nodes. During the 174 hours of data collection, we observed that 6.75
ATLAS jobs were executed on average per hour on Titan and ran for 1.74 hours on
average.

ATLAS jobs issues a large number of file read operations, as can be seen in
Table~\ref{panda-olcf-stats}. The maximum amount read by any ATLAS job in
aggregate in this observed period was less than 250 GB and the maximum amount
written in aggregate was less than 75 GB\@. The average amount read per job is
20 GB and average amount written is 6 GB\@.

\begin{figure}[!htb]
    \centering
    \begin{tabular}{cc}
        {\includegraphics[width=0.48\textwidth]{figures/panda_data_read_finer_hist.pdf}}\\
    \end{tabular}
    \caption{ATLAS file read operation histogram on Titan for week of 10/25/16.}
\label{fig:atlas-titan-io-read}
\end{figure}

\begin{figure}[!htb]
    \centering
    \begin{tabular}{cc}
        {\includegraphics[width=0.48\textwidth]{figures/panda_data_written_finer_hist.pdf}}\\
    \end{tabular}
    \caption{ATLAS file write I/O histogram on Titan for week of 10/25/16.}
\label{fig:atlas-titan-io-written}
\end{figure}

Per job read and write file I/O statistics show an interesting pattern.
Table~\ref{panda-olcf-stats} indicates that the amount of data read per ATLAS
compute node on Titan is less than 400 MB on average, while the amount of data
written per node is less than 170 MB on average. This correlates with our
finding that ATLAS PanDA jobs are read heavy. However, as can be seen in
Table~\ref{panda-olcf-stats} and figures {fig:atlas-titan-io-read} and
{fig:atlas-titan-io-written}, the distribution between read and written amount
of data per job is quite different from one another. The read operation
distribution per job shows a long tail, ranging from 12.5 GB to 250 GB, while
the written amount of data has a very narrow distribution.

\begin{figure}[!htb]
    \centering
    \begin{tabular}{cc}
        {\includegraphics[width=0.48\textwidth]{figures/panda_file_open_hist.pdf}}\\
    \end{tabular}
    \caption{ATLAS file $open()$ histogram on Titan for week of 10/25/16.}
\label{fig:atlas-titan-file-open}
\end{figure}

\begin{figure}[!htb]
    \centering
    \begin{tabular}{cc}
        {\includegraphics[width=0.48\textwidth]{figures/panda_file_close_hist.pdf}}\\
    \end{tabular}
    \caption{ATLAS file $close()$ histogram on Titan for week of 10/25/16.}
\label{fig:atlas-titan-file-close}
\end{figure}

On the metadata I/O breakdown, ATLAS PanDA jobs yield 23 file $open()$
operations and 5 file $close()$ operations per second. The file $open()$
operations listed here don't include the file $stat()$ operations. As can be
seen from figures~\ref{fig:atlas-titan-file-open}
and~\ref{fig:atlas-titan-file-close}, they exhibit similar distributions. The
maximum number of file $open()$ operations are around 170 on average and the
maximum number of file $close()$ operations are 39 on average per second per
job. The total number of file $open()$ operations is 172,089,760 for the
observed window of 174 hours, while the total number of file $close()$
operations stand at 40,132,992 for the total 1,175 ATLAS PanDA jobs in the same
observation window. The difference between these two values is puzzling and it
is under investigation at the time of writing this paper. One possible
explanation is that ATLAS PanDA jobs perhaps don't call a file $close()$
operation per every file $open()$ issued.

\begin{table*}[t]
\centering
\begin{tabular}{lllllllll}
 & Num. Nodes & Duration (s) & Read (GB) & Written (GB) & GB Read/nodes & GB Written/nodes & $open()$ & $close()$ \\
Min & 1 & 1,932 & 0.01 & 0.03 & 0.00037 & 0.02485 & 1,368 & 349 \\
Max & 300 & 7,452 & 241.06 & 71.71 & 0.81670 & 0.23903 & 1,260,185 & 294,908 \\
Average & 35.66 & 6,280.82 & 20.36 & 6.87 & 0.38354 & 0.16794 & 146,459.37 & 34,155.74 \\
Std. Dev. & 55.33 & 520.99 & 43.90 & 12.33 & 0.19379 & 0.03376 & 231,346.55 & 53,799.08
\end{tabular}
\caption{The Statistical breakdown of the I/O impact of 1,175 PanDA jobs executed at OLCF for the week of 10/25/16}
\label{panda-olcf-stats}
\end{table*}

Overall, based on our experiments with real-world jobs, it can be safely
concluded that the file and metadata I/O load of ATLAS PanDA project on the OLCF
Titan supercomputing environment and the Spider 2 file system is not detrimental
to the center operations and the overall impact is minimal, at the current scale
of the project.


%\section{Putting it all together: Experience of PanDA on TITAN} % (fold)
%\label{sec:panda_titan}
%
%\begin{enumerate}
%  \item Metadata performance issue, and how it’s resolved
%  \item File I/O performance and impact
%  \item ?
%\end{enumerate}


% ----------------------------------------------------------------------------
% VII - THE NEXT GENERATION EXECUTOR
% ----------------------------------------------------------------------------
% \section{PANDA: Future/RoadMap}
\section{PANDA\@: The Next Generation Executor}
\label{sec:panda_roadmap}

\begin{enumerate}
    \item Discussion of the limitations of the current state of the art
    \item Rationale and design (why)
    \item Architecture (how)
    \item Integration
    \item Characterization (experiments)
\end{enumerate}

\subsection{Experiments}
\label{sec:ngeExp}

We designed experiments to characterize the performance of the NGE on Titan,
with an emphasis on understanding its overhead and thus the cost of introducing
new functionalities.  We perform three groups of experiments in which we
investigate the weak scalability, weak scalability with multiple generation, and
strong scalability of the NGE.

Each experiment entails executing multiple instances of AthenaMP using the NGE
to simulate a pre-determined number of events. All the experiments have been
performed by  configuring AthenaMP to use all the 16 cores  of Titan's worker
nodes.

We  measured the execution time of the pilots and of the AthenaMP  executed
within them, collecting timestamps at  all stages of the execution. Experiments
have been performed  by  submitting NGE's pilots  to Titan's batch queue.  The
turnaround time of an individual run is determined by queue waiting times. Since
we are interested only in the performances of the NGE, we removed queue time
from our statistics.

\subsubsection{Weak scalability}

In this experiment  we run as many AthenaMP instances (hereafter referred to as
tasks)  as the number of nodes controlled by the pilot. Each AthenaMP simulates
100 events, requiring $\sim 4200$ seconds on average.

Tasks do not  wait within the NGE Agent's queue since  one node  is available to
each AthenaMP instance.  Overheads in task execution are consequence primarily
of the three other factors: (i) the  initial bootstrapping of the pilot on the
nodes; (ii) the UnitManager's dispatching of  units (tasks) to the agent; and
(iii) time for the agent to bootstrap all the tasks on the nodes.

We tested  pilots  with 250, 500, 1000 and 2000 worker nodes and 2 hours
walltime. The time duration is determined by the Titan's walltime policy.
Fig.~\ref{fig:weakScal1a} depicts the average pilot duration, the average
execution time of AthenaMP, and the NGE pilot overhead as function of the pilot
size.

\begin{figure}[!htb]
        \includegraphics[height=4.5cm,width=\columnwidth]{./figures/NGE/weak1.pdf}
    \caption{Weak scalability: average pilot duration, average  duration of a
    single AthenaMP execution, and pilot's overhead as a function of different pilot sizes (200, 500, 1000 and 2000 nodes).}
\label{fig:weakScal1a}
\end{figure}

We observe that, despite some fluctuations due to external factors (e.g.,
Titan's shared filesystem and the shared database used by the NGE), the average
execution time of AthenaMP  ranges between 4200 and 4800 seconds.  We  also
observe that in all the cases the gap between AthenaMP execution times and the
pilot durations is minimal, although it slightly increases with the pilot size.
We  notice that NGE's overhead does grow linearly with the number of units.

\subsubsection{Weak scalability with multiple generation }

The NGE provides an important new capability of submitting multiple generations
of AthenaMP to the same pilot. In order to investigate the cost of doing so, we
performed a variant of the weak scalability experiments. This stresses the
pilot's components, as new tasks are scheduled for execution on the Agent while
other tasks are still running.

In these experiments, we run five AthenaMP instances per node.  As these
experiments are designed to investigate the overhead generated by the scheduling
and bootstrap of AthenaMP instances, we reduced the number of events simulated
by each AthenaMP task to sixteen in such a way that the running time of each
AthenaMP is, on average, $\sim 1200$ seconds. This experiment design choice does
not affect the  objectives or accuracy of the experiments, but allows us to
scale experiments to large node counts by being conservative with allocation.

We ran pilots with 256, 512, 1024 and 2048 worker nodes and 3\mtnote{2?} hours
walltime. Fig.~\ref{fig:weakScal2a} depicts the average pilot duration, the
average execution time of five sequential generations of AthenaMP, and the
corresponding overhead. We observe that the difference between the two durations
is more marked than in the previous experiments. Despite this, we notice that
the growth of the overhead is consistent with the increment of the number of
tasks per node for pilots with 256, 512 and 1024 worker nodes, and less than
linear for the pilot with 2048 worker nodes.

\begin{figure}[!htb]
        \includegraphics[height=4.5cm,width=\columnwidth]{./figures/NGE/weak2.pdf}
    \caption{Weak scalability with multiple generations: average pilot
    duration, average duration of five sequential AthenaMP executions, and
    pilot's overhead for different pilot sizes (256, 512, 1024 and 2048 nodes).}
\label{fig:weakScal2a}
\end{figure}

\subsubsection{Strong scalability}

The last experiments  study strong scalability by running the same number of
tasks for different pilot sizes. We used 2048 AthenaMP instances and  pilots
with 256, 512, 1024 and 2048 nodes. Thus, the number of AthenaMP generations is
equal to eight times the size of the smallest pilot and corresponds to the size
of the largest pilot. As a consequence, the number of consecutive generations of
AthenaMP decreases with the pilot size by generating different dynamics within
the pilots. These experiments are designed to investigate whether pilot overhead
is affected by the degree of concurrency within the pilot and/or the number of
tasks. Each AthenaMP instance simulates sixteen events as in the previous
experiment.

Fig.~\ref{fig:strongScala}  shows the average pilot duration and the average
execution time of possibly sequential AthenaMP instances.  We  notice that the
difference between the pilot duration and the AthenaMP execution times is almost
constant for all the pilot sizes, although the overall duration of the pilot
decreases linearly with the pilot size.

\begin{figure}[!htb]
        \includegraphics[height=4.5cm,width=\columnwidth]{./figures/NGE/strong.pdf}
    \caption{Strong scalability:  average pilot duration, average duration of
    sequential AthenaMP executions, and pilot's overhead for different pilot
    sizes (256, 512, 1024 and 2048 nodes).}
\label{fig:strongScala}
\end{figure}



% ----------------------------------------------------------------------------
% VIII - CONCLUSIONS
% ----------------------------------------------------------------------------
\section{Conclusion}
\label{sec:conclusion}

The PanDA  system was developed to meet the scale and complexity of LHC
distributed computing for the ATLAS experiment.  In the process,  the old batch
job paradigm  of computing in HEP was  discarded  in favor of a  far more
flexible and scalable  model. The success  of PanDA  at the LHC is leading to
widespread adoption and testing by other experiments. PanDA  is the first
exascale  workload management system in HEP, already operating at a million
computing jobs per day, and processing over an exabyte of data in 2013. Next LHC
run will pose massive computing  challenges. With a  doubling of the beam
energy  and luminosity as  well as an increased  need for  simulates  data, the
data volume is expected to increase with a factor 5--6 or more. Storing and
processing  this amount of data is a  challenge   that cannot be resolved with
the currently existing  computing  resources in ATLAS\@. To resolve this
challenge, ATLAS is turning to commercial  as well as academic Cloud services
and HPCs via the PanDA system. Also the work underway is enabling the use of
PanDA by new scientific collaborations and communities as a means  of leveraging
extreme scale computing  resources with a low barrier of entry. The technology
base provided by the PanDA system will enhance the usage of a variety  of
high-performance computing resources available to basic research.


% ----------------------------------------------------------------------------
% ACKNOWLEDGEMENTS
% ----------------------------------------------------------------------------
This research used resources of the Oak Ridge Leadership Computing Facility,
supported by the Office of Science of the Department of Energy under Contract
DE-AC05-00OR22725.


%\section*{Acknowledgment}

%The authors would like to thank... More thanks here...

% trigger a \newpage just before the given reference
% number - used to balance the columns on the last page
% adjust value as needed - may need to be readjusted if
% the document is modified later
%\IEEEtriggeratref{8}
% The "triggered" command can be changed if desired:
%\IEEEtriggercmd{\enlargethispage{-5in}}

% references section

% can use a bibliography generated by BibTeX as a .bbl file
% BibTeX documentation can be easily obtained at:
% http://www.ctan.org/tex-archive/biblio/bibtex/contrib/doc/
% The IEEEtran BibTeX style support page is at:
% http://www.michaelshell.org/tex/ieeetran/bibtex/
%\bibliographystyle{IEEEtran}
% argument is your BibTeX string definitions and bibliography database(s)
%\bibliography{IEEEabrv,../bib/paper}
%
% <OR> manually copy in the resultant .bbl file
% set second argument of \begin to the number of references
% (used to reserve space for the reference number labels box)

% ----------------------------------------------------------------------------
% REFERENCES
% ----------------------------------------------------------------------------
\bibliographystyle{plain}
\bibliography{bibliography}

% that's all folks
\end{document}
