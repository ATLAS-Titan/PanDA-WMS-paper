% Experiments at the Large Hadron Collider (LHC) face unprecedented computing
% challenges. Heterogeneous resources are distributed worldwide, thousands of
% physicists analyzing the data need remote access to hundreds of computing
% sites, the volume of processed data is beyond the exabyte scale, and data
% processing requires more than billions of hours of computing usage per year. The
% PanDA (Production and Distributed Analysis) system was developed to meet the
% scale and complexity of LHC distributed computing for the ATLAS experiment. In
% the process, the old batch job paradigm of computing in HEP was discarded in
% favor of a far more flexible and scalable model. The success of PanDA at the LHC
% is leading to widespread adoption and testing by other experiments. PanDA is
% the first exascale workload management system in HEP, already operating at
% a million computing jobs per day, and processing over an exabyte of data in
% 2013. We will describe the design and implementation of PanDA, present data on
% the performance of PanDA at the LHC, and discuss plans for future evolution
% of the system to meet new challenges of scale, heterogeneity and increasing
% user base.

% Experiments at the Large Hadron Collider (LHC) face unprecedented computing
% challenges. Thousands of physicists analyze exabytes of data every year, using
% billions of computing hours on hundreds of computing sites worldwide. PanDA
% (Production and Distributed Analysis) is a workload management system (WMS)
% developed to meet the scale and complexity of LHC distributed computing for the
% ATLAS experiment. PanDA is the first exascale workload management system in HEP,
% executing millions of computing jobs per day, and processing over an exabyte of
% data in 2016. In this paper, we introduce the design and implementation of
% PanDA, describing its deployment on Titan, the third biggest supercomputer in
% the world. We analyze scalability, reliability and performance of PanDA on
% Titan, highlighting the challenges addressed by its architecture and
% implementation. We present preliminary results of experiments performed with the
% Next Generation Executer, a prototype we developed to meet new challenges of
% scale and resource heterogeneity.

% As the LHC prepares for Run 3 in
% $~\approx$ 2022 and the high-luminosity era (Run 4), it is anticipated that
% the data volumes that will need analyzing will increase by factors of 10-100
% compared to the current phase (Run 2). Data will be larger in volume but will
% also require more sophisticated computational processing. 

The computing systems used by LHC experiments has historically consisted of
the federation of hundreds to thousands of distributed resources, ranging from
small to mid-size resource.  In spite of the impressive scale of the existing
distributed computing solutions, the federation of small to mid-size
resources will be insufficient to meet projected future demands.
This paper is a case study of how the ATLAS experiment has embraced Titan -- a
DOE leadership facility in conjunction with traditional distributed high-
throughput computing to reach sustained production scales of approximately 51M
core-hours a years. The three main contributions of this paper are:  (i) a
critical evaluation of design and operational considerations  to support the
sustained, scalable and production usage of Titan;  (ii) a preliminary
characterization of a next generation executor for PanDA to support new
workloads and  advanced execution modes; and (iii) early lessons for how
current and future experimental and observational systems can be integrated
with production supercomputers and other platforms in a general and extensible
manner.
