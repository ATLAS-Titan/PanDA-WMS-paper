The computing systems used by LHC experiments has historically consisted of
the federation of hundreds to thousands of distributed resources, ranging
from small to mid-size resource.  In spite of the impressive scale of the
existing distributed computing solutions, the federation of small to mid-size
resources will be insufficient to meet projected future demands. This paper
is a case study of how the ATLAS experiment has embraced Titan -- a DOE
leadership facility in conjunction with traditional distributed high-
throughput computing to reach sustained production scales of approximately
52M core-hours a years. The three main contributions of this paper are:  (i)
a critical evaluation of design and operational considerations  to support
the sustained, scalable and production usage of Titan;  (ii) a preliminary
characterization of a next generation executor for PanDA to support new
workloads and  advanced execution modes; and (iii) early lessons for how
current and future experimental and observational systems can be integrated
with production supercomputers and other platforms in a general and
extensible manner.