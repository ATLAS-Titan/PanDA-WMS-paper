\ifreview
Comments to address in this section:
\begin{enumerate}
	\item REVIEWER 1: listing the key observations in one place, and in a
	consistent fashion. I found The 5 points listed in the first part of the
	rebuttal quite compelling, and strongly urge the authors to put them in
	the paper in a prominent location.
	\item REVIEWER 1: A glossary would be welcomed.
	\item REVIEWER 5: I would have preferred to see a more detailed list of
	contributions, including the novelty of the proposed approach as compared
	to prior work.
	\item I am ambivalent about the use of the term ``converging'' in the
	paper's title. While the paper does port workloads to high-performance
	computing resources, it does not deal with compute-oriented optimizations
	(parallelization), lowering communication, etc., which are often major
	concerns in HPC programming. In this sense, the paper is closer in spirit
	to how to port high-throughput apps to high-performance clusters. Agreed?
\end{enumerate}
\fi



The Large Hadron Collider (LHC) was created to explore the fundamental
properties of matter. Multiple experiments at LHC have collected and
distributed hundreds of petabytes of data worldwide to hundreds of computer
centers. Thousands of physicists analyze petascale data volumes daily. The
detection of the Higgs Boson in 2013 speaks to the success of the detector
and experiment design, as well as the sophistication of computing systems
devised to analyze the data.

Historically, the computing systems used by LHC experiments consisted of the
federation of hundreds to thousands of distributed resources, %\textemdash{}
ranging in scale from small to mid-size resource~\cite{foster2003grid}.
Although the workloads to be executed are comprised of tasks that are
independent of each other, the management of the distribution of workloads
across many heterogeneous resources, the effective utilization of resources
and efficient execution of workloads presents non-trivial challenges.

Many software solutions have been developed in response to these challenges.
CMS, one of the LHC experiments, devised a solution based around the
HTCondor~\cite{thain2005distributed} software ecosystem. The
ATLAS~\cite{Aad:2008} experiment, utilizes the Production and Distributed
Analysis (PanDA) workload management system~\cite{Maeno2011} (WMS) for
distributed data processing and analysis. The CMS and ATLAS experiments
represent arguably the largest production grade distributed computing
solutions and have symbolized the paradigm of {\it high-throughput
computing}, i.e., the effective execution of many independent tasks.

As the LHC prepares for Run 3 in $~\approx$ 2022 and the high-luminosity era
(Run 4), it is anticipated that the data volumes that will need analyzing
will increase by factors of 10-100 compared to the current phase (Run 2).
Data will be larger in volume but will also require more sophisticated
computational processing. In spite of the impressive scale of the ATLAS
distributed computing system, demand for computing systems will significantly
outstrip current and projected supply.  There are multiple levels at which
this problem needs to be addressed: the utilization of emerging parallel
architectures (e.g., platforms); algorithmic and advances in analytical
methods (e.g., use of Machine Learning); and the ability to exploit different
platforms (e.g., clouds and supercomputers).

This paper represents the experience of how the ATLAS experiment has ``broken
free'' of the traditional computational approach of high-throughput computing
on distributed resources to embrace new platforms, in particular  high-
performance computers. Specifically, we discuss the experience of integrating
PanDA WMS with a US DOE leadership computing facility called Titan to reach
sustained production scales of approximately 51M core-hours a year. We also
discusses the investigation of a pilot-abstraction based task execution
runtime system on Titan, which allows advanced execution modes of the ATLAS
workload as well as the enhanced support for heterogeneous workloads, e.g.,
molecular dynamics.

% . Consequently, Titan now analyzes up to 3.6\% of the simulation workload
% of the ATLAS experiment. The case study presents the design and integration
% of PanDA

This experience paper provides three main contributions:  (i) a critical
evaluation of the many design and operational considerations that have been
taken to support the sustained, scalable and production usage of Titan for
historically high-throughput workloads; (ii) a preliminary characterization
of a next generation executor (NGE) for PanDA to support non-traditional
heterogeneous workloads and execution modes;  and (iii) early lessons and
guidance as the community looks forward to designing the next generation of
online analytical platforms~\cite{foap-url}, for how current and future
experimental and observational systems can be integrated with production
supercomputers in a general and extensible manner.
