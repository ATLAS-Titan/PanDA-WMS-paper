The Large Hadron Collider (LHC) was created to explore the fundamental
properties of matter for the next decades.   Multiple experiments at LHC have
collected and distributed hundreds of petabytes of data worldwide to hundreds
of computer centers. Thousands of physicists analyze petascale data volumes
daily. The detection of the Higgs Boson in 2013 speaks to the success of the
detector and experiment design, as well as the sophistication of computing
systems devised to analyze the data.

Historically the computing systems used by LHC experiments consisted of the
federation of a hundreds to thousands of distributed resources \textemdash{}
ranging in scale from small to mid-size resource~\cite{foster2003grid}.
Although the workloads to be executed are comprised of tasks that are
independent of each other, the management of the distribution of 
workloads across many heterogeneous resources to ensure the effective
utilization of resources and efficient execution of workloads presents non-
trivial challenges.

Many software solutions have been developed in response to these challenges.
One of the LHC experiments, the CMS experiment devised a solution based around
the HTCondor~\cite{XX} software ecosystem. The ATLAS~\cite{Aad:2008}
experiment, utilizes the Production and Distributed Analysis (PanDA) workload
management system~\cite{Maeno2011} (WMS) for distributed data processing and
analysis. The CMS and ATLAS experiments represent arguably the largest
production grade distributed computing solutions and have symbolized the
paradigm of {\it high-throughput computing} viz., the effective execution of
many independent tasks.

As the LHC prepares for the high-luminosity era (Run 4) and Run 3 in
$~\approx$ 2022, it is anticipated that the data volumes that will need
analyzing will increase by factors of 10-100  compared to the current phase
(Run 2). The data will be larger in volume but will also require more
sophisticated computational processing. In spite of the impressive scale of
the ATLAS distributed computing system, demand for computing systems will
significantly outstrip supply (availability).

There are multiple levels at which this problem needs to be addressed
urgently, e.g., the utilization of emerging parallel architectures (e.g.,
platforms), algorithmic and advances in analytical methods (e.g., use of
Machine Learning) and the ability to exploit different platforms (e.g., clouds
and supercomputers).

This paper is a case study of how the ATLAS experiment has "broken free" of
the traditional computational approach of high-throughput computing on
distributed resources to embrace new platforms, in particular high-performance
computers. Specifically, we discuss the experience of integrating the PanDA
workload management system with Titan ~\textemdash{} a DOE leadership
computing facility. Consequently, Titan now analyzes up to 5\% of the
simulation workload  of the ATLAS experiment. The case study presents the
design and integration of PanDA to reach sustained production scales (ca. 100
M core-hours a years). It also discusses the investigation of a task execution
runtime system on Titan, that is based on  the pilot-abstractions and which
allows advanced execution modes of the ATLAS workload as well as the enhanced
support for heterogeneous workloads (such as molecular dynamics).

This state-of-practice paper provides multiple contributions.  It (i)
documents the many design and operational considerations that have been  taken
to support the sustained, scalable and production usage of Titan for
historically high-throughput workloads, (ii) Extensions to PanDA to support
non-traditional heterogeneous workloads and execution modes,  and (iii) As the
community looks forward to designing the next generation of online analytical
platforms~\cite{foap}, this project provides some early lessons and guidance
for how current and future experimental and observational systems can be
integrated with production supercomputers in a general and extensible manner.


% \subsubsection*{The following is a candidate for major compression into a single paragraph}

% \mtnote{Used the following in related work. We may want to reduce/remove it from
% the introduction: ``Today, PanDA serves several thousand users, managing job
% distribution to hundreds of ATLAS sites with more than 100,000 CPU cores which
% process more than a million jobs per day.''} % (Figure~\ref{fig:daily}).

% In this paper, we describe how PanDA has been engineered to execute a specific
% stage of the ATLAS Monte Carlo workflow on Titan, the larger high-performance
% computing HPC system currently available in the USA\@.\mtnote{Explain the
% benefits offered by Titan in terms of multithreading per node and possibly large
% amount of concurrent nodes. Introduce also the notion of backfill.} This extends
% the scope of PanDA's compute model, integrating both high-throughput and
% high-performance computing resources and enabling the concurrent execution of
% both  single and multi-core jobs. The integration of PanDA and Titan went
% through three main engineering phases: (i) feasibility study and rapid
% prototyping of an initial solution; (ii) progressing scaling of the  prototype
% to saturate the available resources; (iii) study of a product-grade architecture
% for generic HPC resources. Both phase i and ii have been completed enabling the
% execution of up to eight million jobs a week on Titan. A prototype has been
% engineered to support phase iii and experimental data are being collected.

% In the next section we introduce \ldots.

% Why Titan?
% \begin{itemize}
%     \item A lot of slow (relative the grid) and homogeneous cores together
%     \item Grid is saturated
%     \item Enable HPC as a calss of computing resource
%     \item Enable future DOE experimental and observational capabilities on HPC
%     \item Why is so important: more data, run 2 and run 3.
%     \item growth of grid is flat (economic model) and saturated. We need more CPU because we have more data.
%     \item ATLAS spend most of the time on simulations this is why we want to offload simulations - that happen happen to be performed via AthenaMP.
% \end{itemize}


% \mtnote{Moved from related work were we use already this terminology. To be
% iterated/adjusted for consistency once the first draft will be ready.}

% The term ``workflow'' is used in many disciplines with different meaning. In the
% field of scientific computing, ``workflow'' assumes different meanings depending
% on the characteristics of the computation, of the software tools used to support
% this computation, and of the resources on which it is performed. Further, a
% workflow may indicate a whole application, a description of the computational
% process of that application or, more commonly, a series of tasks related by data
% dependences.

% The lack of a consistent and shared definition of ``workflow'' hinders the
% understanding of its properties and its relations with related concepts. For
% example, we need to clarify the difference among ``workflow'', ``workload'',
% ``task'', or ``job'' but also between workflow ``template'' and ``instance'', or
% ``data-flow'' and ``control-flow''. This is precondition to specify properly the
% design of software systems that support the execution of scientific
% applications.

% In this paper, we use the following definitions:

% \begin{description}

%   \item[Task.] A set of operations to be performed on a computing platform,
%   alongside a description of the properties and dependences of those operations,
%   and indications on how the dependences should be satisfied and the operations
%   should be executed.

%   \item[Job.] A unit of  work performed by submitting a script to a resource
%   management system (LRMS), like  Slurm or PBS, or by requesting a virtual
%   machine or a container to a site supporting virtualization. One or more jobs
%   can perform the operations described with a task.

%   \item[Workload.] A set of jobs that can be executed concurrently, possibly
%   related by a set of relations. For example, jobs of a workload can share one
%   or more input files or communicate during execution.

%   \item[Workflow.] Set of jobs, related by a set of relations that define the
%   order in which each task can be executed. Data dependences are the most common
%   relations among workloads, used to define the precedence among their
%   executions.

% \end{description}
